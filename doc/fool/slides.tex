\documentclass{beamer}

%%%%%%%%%%%%%%%%%%%%%%%%%%%%%%%%%%%%%%%
%   Language abstraction commands     %
%%%%%%%%%%%%%%%%%%%%%%%%%%%%%%%%%%%%%%%

%% Relations
% Subtype 
\newcommand{\sub}{<:}
% Type assignment
\newcommand{\typ}{:}
% reduction
\newcommand{\reduces}{\;\rightarrow\;}
% well-formedness
\newcommand{\wf}{\;\mbox{\textbf{wf}}}
\newcommand{\nswf}{\mbox{\textbf{wf}}}
\newcommand{\wfe}{\;\mbox{\textbf{wfe}}}
\newcommand{\nswfe}{\mbox{\textbf{wfe}}}

%% Operators
% Type selection
\newcommand{\tsel}{\#}
% Function type
\newcommand{\tfun}{\rightarrow}
\newcommand{\dfun}[3]{(#1\!:\!#2) \Rightarrow #3}
% Conjunction
\newcommand{\tand}{\wedge}
% Disjunction
\newcommand{\tor}{\vee}
% Singleton type suffix
\newcommand{\sing}{.\textbf{type}}

%% Syntax
% Header for typing rules
\newcommand{\judgement}[2]{{\bf #1} \hfill #2}
% Widening
\newcommand{\wid}[2]{#1 : #2}
% Refinement
\newcommand{\refine}[2]{\left\{#1 \Rightarrow #2 \right\}}
\newcommand{\mlrefine}[2]{\{#1 \Rightarrow #2 \}}
% Field definitions
\newcommand{\ldefs}[1]{\left\{#1\right\}}
\newcommand{\mlldefs}[1]{\{#1\}}
% Member sequences
\newcommand{\seq}[1]{\overline{#1}}
% Lambda
\newcommand{\dabs}[3]{(#1\!:\!#2)\Rightarrow #3}
\newcommand{\abs}[3]{\lambda #1\!:\!#2.#3}
% Method Application
\newcommand{\mapp}[3]{#1.#2(#3)}
% Substitution
\newcommand{\subst}[3]{[#1/#2]#3}
% Object creation
\newcommand{\new}[3]{\textbf{val }#1 = \textbf{new }#2 ;\; #3}
\newcommand{\mlnew}[3]{\textbf{val }#1 = \textbf{new }#2 ;\;\\&#3}
%\renewcommand{\new}[3]{#1 \leftarrow #2 \,\textbf{in}\, #3}
% Field declaration
\newcommand{\Ldecl}[3]{#1 : #2..#3}%{#1 \operatorname{>:} #2 \operatorname{<:} #3}
\newcommand{\ldecl}[2]{#1 : #2}
\newcommand{\mdecl}[3]{#1 : #2 \tfun #3}
% Top and Bottom
\newcommand{\Top}{\top}%{\textbf{Top}}
\newcommand{\Bot}{\bot}%\textbf{Bot}}
% Environment extension
%\newcommand{\envplus}[1]{\uplus \{ #1 \}}
\newcommand{\envplus}[1]{, #1}
% Reduction
\newcommand{\reduction}[4]{#1 \operatorname{|} #2 \reduces #3 \operatorname{|} #4}

% Sugar
\newcommand{\arrow}[2]{#1\rightarrow_s#2}
\newcommand{\fun}[4]{\textbf{fun } (#1:#2)\;#3\;#4}
\newcommand{\app}[2]{(\textbf{app }#1\;#2)}
\newcommand{\mlapp}[2]{(\textbf{app }#1\;\\&#2)}
\newcommand{\cast}[2]{(\textbf{cast }#1\;#2)}

\newcommand{\lindent}{\hspace{-4mm}}

% Logical relations
\newcommand{\relv}[4]{\mathcal{V}_{#1;#2;#3}\llbracket#4\rrbracket}
\newcommand{\rele}[4]{\mathcal{E}_{#1;#2;#3}\llbracket#4\rrbracket}
\newcommand{\rels}[3]{\mathcal{\supseteq}_{#1}\llbracket#2;#3\rrbracket}
\newcommand{\relg}[3]{\mathcal{\supseteq^!}_{#1;#2}\llbracket#3\rrbracket}
\newcommand{\irred}[2]{\text{irred }(#1,#2)}
\newcommand{\andl}{\;\wedge\;}
\newcommand{\orl}{\vee}
\newcommand{\impliesl}{\rightarrow}
\newcommand{\reductionl}[5]{#1 \operatorname{|} #2 \;\rightarrow^{#5}\; #3 \operatorname{|} #4}
\newcommand{\ds}{\,\vDash\,}


\usepackage{minted}
\usemintedstyle{eclipse}

\useoutertheme{infolines}
\setbeamertemplate{headline}{}
\setbeamertemplate{footline}{
  \hfill
  \usebeamercolor[fg]{page number in head/foot}
  \usebeamerfont{page number in head/foot}
  \insertpagenumber\kern1em\vskip10pt
}
\setbeamertemplate{navigation symbols}{}

\title{Dependent Object Types}
\subtitle{Towards a foundation for Scala's type system}
\author{Nada Amin, Adriaan Moors, Martin Odersky}
\institute{FOOL 2012}
\date{October 22, 2012}

\begin{document}

\frame{\titlepage}

\section{Introduction}

\begin{frame}
\frametitle{DOT: Dependent Object Types}

The DOT calculus proposes a new \emph{type-theoretic foundation} for Scala
and languages like it. It models
\begin{itemize}
\item path-dependent types
\item abstract type members
\item mixture of nominal and structural typing via refinement types
\end{itemize}

It does not model
\begin{itemize}
\item inheritance and mixin composition
\item what's currently in Scala
\end{itemize}

DOT normalizes Scala's type system by
\begin{itemize}
\item unifying the constructs for type members
\item providing classical intersection and union types
\end{itemize}

\end{frame}

\section{Differences with Scala}

\begin{frame}[fragile]{Classicial Intersection and Union Types}
\begin{itemize}
\item form a lattice wrt subtyping
\item simplify glb and lub computations
\end{itemize}
\begin{minted}{scala}
  trait A { type T <: A }
  trait B { type T <: B }
  // in Scala, glb is an infinite sequence
  A with B { type T <: A with B { type T <: A with B {
    type T <: ...
  }}}
  // in DOT, glb is simply A /\ B
  // type inference needs to compute glbs and lubs
  if (cond) ((a: A) => c: C) else ((b: B) => d: D)
  // glb(A, B) => lub(C, D)
\end{minted}

\end{frame}

\begin{frame}[fragile]{Constructs for Type Memers}
\begin{minted}{scala}
trait Food
trait Animal {
  // in DOT, Meal: Bot .. Food
  type Meal <: Food
  def eat(meal: Meal) {}
}

trait Grass extends Food
trait Cow extends Animal {
  // in DOT, Meal: Grass .. Grass
  type Meal = Grass
}
\end{minted}
\end{frame}

\section{Formalities}

\begin{frame}[fragile]{DOT: Syntax}
\begin{itemize}
\item terms
\begin{description}[method invocations]
\item[variables] $x$, $y$, $z$
\item[selections] $t.l$
\item[method invocations] $t.m(t)$
\item[object creations] $\new y c {t'}$\\
$c$ is a constructor $T_c \ldefs{\seq{l = v}\;\seq{m(x) = t}}$
\end{description}
\item types
\begin{description}[type intersections]
\item[type selections] $p.L$
\item[refinement types] $T \refine z {\seq D}$
\item[type intersections] $T \tand T'$
\item[type unions] $T \tor T'$
\item[a top type] $\Top$
\item[a bottom type] $\Bot$
\end{description}
\end{itemize}
\end{frame}

\begin{frame}[fragile]
\frametitle{DOT: Judgments}
\begin{columns}
\begin{column}[t]{5cm}
\begin{block}{Typing Judgments}
\begin{itemize}
\item type assignment\\$\Gamma \ts t \typ T$
\item subtyping\\$\Gamma \ts S \sub T$
\item well-formedness\\$\Gamma \ts T \wf$
\item membership\\$\Gamma \ts t \ni D$
\item expansion\\$\Gamma \ts T \expand_z \seq{D}$
\end{itemize}
\end{block}
\end{column}
\begin{column}[t]{5cm}
\begin{block}{Small-Step Operational Semantics}
\begin{itemize}
\item reduction\\$\reduction t s {t'} {s'}$
\end{itemize}
\end{block}
\end{column}
\end{columns}
\end{frame}

\begin{frame}[fragile]
\frametitle{Example: Functions}
\begin{align*}
\arrow S T &\iff {\Top \refine z {\mdecl {\mi{apply}} S T}}\\
\fun x S T t &\iff \new z {{\arrow S T}\ldefs{\mi{apply}(x) = t}} z\\
\app f x &\iff \mapp f {\mi{apply}} x\\
\cast T t&\iff \app {(\fun x T T x)} t
\end{align*}
\end{frame}


\begin{frame}[fragile]
\frametitle{Example: Class Hierarchies}
\begin{minted}[fontsize=\footnotesize]{scala}
object pets {
  trait Pet
  trait Cat extends Pet
  trait Dog extends Pet
  trait Poodle extends Dog
  trait Dalmatian extends Dog
}
\end{minted}
\begin{align*}
\mlnew {&\mi{pets}} {\Top \mlrefine z {\\
&\ \Ldecl {\mi{Pet_c}} \Bot \Top\\
&\ \Ldecl {\mi{Cat_c}} \Bot {z.{\mi{Pet_c}}}\\
&\ \Ldecl {\mi{Dog_c}} \Bot {z.{\mi{Pet_c}}}\\
&\ \Ldecl {\mi{Poodle_c}} \Bot {z.{\mi{Dog_c}}}\\
&\ \Ldecl {\mi{Dalmatian_c}} \Bot {z.{\mi{Dog_c}}}\\
} \ldefs{}}{}
\end{align*}
\end{frame}

\begin{frame}[fragile]
\frametitle{Example: Abstract Type Members}
\begin{minted}[fontsize=\footnotesize]{scala}
object choices {
  trait Alt {
    type C
    type A <: C
    type B <: C
    val choose : A => B => C
  }
}
\end{minted}
\begin{align*}
&\new {\mi{choices}} {\Top \mlrefine z {\\
&\gap\Ldecl {\mi{Alt_c}} {\Bot} {\Top \mlrefine a {\\
&\gap\gap\Ldecl C \Bot \Top\\
&\gap\gap\Ldecl A \Bot {a.C}\\
&\gap\gap\Ldecl B \Bot {a.C}\\
&\gap\gap\mdecl {\mi{choose}} {a.A} {\arrow {a.B} {a.A \tor a.B}}\\
&\gap}}\\
&}\ldefs{}}{}
\end{align*}
\end{frame}

\begin{frame}[fragile]
\frametitle{Example: F-bounded Quantification}
\begin{minted}[fontsize=\footnotesize]{scala}
trait MetaAlt extends choices.Alt {
  type C = MetaAlt
  type A = C
  type B = C
}
\end{minted}
\begin{align*}
&\mlnew m {\Top \mlrefine m {\\
&\gap\Ldecl {\mi{MetaAlt_c}} {\Bot} {{\mi{choices.Alt_c}} \mlrefine a {\\
&\gap\gap\Ldecl C {\mi{m.MetaAlt_c}} {\mi{m.MetaAlt_c}}\\
&\gap\gap\Ldecl A {\mi{a.C}} {\mi{a.C}}\\
&\gap\gap\Ldecl B {\mi{a.C}} {\mi{a.C}}\\
&\gap}}\\
&}\ldefs{}}{}
\end{align*}
\end{frame}

\end{document}
