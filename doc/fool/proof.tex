\section{Type Safety via Logical Relations}\label{dot-type-safety}
We sketch a proof of type-safety of the DOT calculus via step-indexed
logical relations~\cite{ahmed04,ahmed06,step_indexed_obj}.

\subsection{Type Safety}
Type-safety states that a well-typed program doesn't get stuck. More
formally: If $\emptyset \ts t \typ T$ and $\reductionl t \emptyset {t'} {s'} *$ then
either $t'$ is a value or $\exists t'', s''. \reduction {t'} {s'}
{t''} {s''}$.

Our strategy is to define a logical relation $\Gamma \ds t : T$, such
that $\Gamma \ts t : T$ implies $\Gamma \ds t : T$ implies
type-safety.

\subsection{Step-Indexed Logical Relations}

In order to ensure that our logical relation is well-founded, we use a
step index. For each step index $k$, we define the set of values and
the set of terms that appear to belong to a given type, when taking at
most $k$ steps. $\Gamma \ds t : T$ is then defined in terms of the
step-indexed logical relation by requiring it to hold $\forall k$.

\subsubsection{Set of Values}
$\relv k \Gamma s T$ defines the set of values that appear to have
type $T$ when taking at most $k$ steps. $\Gamma$ and $s$ must agree:
$\dom(\Gamma) = \dom(s)$ (ordered) and $\forall (x : T) \in \Gamma, x
\in \relv k \Gamma s T$. A variable $y$ belongs to $\relv 0 \Gamma s
T$ simply by being in the store. In addition, it belongs to $\relv k
\Gamma s T$ for $k > 0$, if it defines all type, method and value labels in
the expansion of $T$ appropriately for $j < k$ steps.
\begin{align*}
&\relv k \Gamma s T = \{ y \ | y \in \dom(s) \andl (\\
& (\Gamma \ts T \wfe \andl\\
&\gap \forall j < k,\\
&\gap y \mapsto T_c \ldefs{\seq{l = v}\;\seq{m(x)=t}} \in s,\\
&\gap \Gamma \ts T \expand_y \seq{D},\\
&\gap (\forall L_i : S \rightarrow U \in \seq{D},\\
&\gap\gap \Gamma \ts y \ni L_i \typ {S'..U'}) \andl\\
&\gap (\forall m_i : S \rightarrow U \in \seq{D},\\
&\gap\gap t_i \in \rele j {\Gamma \envplus{x_i : S}} s U) \andl\\
&\gap (\forall l_i : V \in \seq{D},\\
&\gap\gap v_i \in \relv j \Gamma s V)) \orl\\
& (T = T_1 \tand T_2 \andl y \in \relv k \Gamma s {T_1} \andl y \in \relv k \Gamma s {T_2}) \orl\\
& (T = T_1 \tor T_2 \andl  (y \in \relv k \Gamma s {T_1} \orl y \in \relv k \Gamma s {T_2}))\\
&)\}
\end{align*}

This relation captures the observation that the only ways for a term
to get stuck is to have a field selection on an uninitialized field or
a method invocation on an uninitialized method. However, a {\it potential
  pitfall} is that the value itself might occur in the types $S$, $U$,
$V$, because we substitute it for the ``self'' occurrences in the
expansion, so the relation makes sure that the required type labels exist.

\subsubsection{Set of Terms}
$\rele k \Gamma s T$ defines the set of terms that appear to have type
$T$ when taking at most $k$ steps. $s$ must agree with a {\it prefix}
of $\Gamma$, so $\Gamma$ can additionally contain variables not in
$s$. This is needed for checking methods in $\mathcal{V}$ above, and
for relating open terms. If $k > 0$, $\mathcal{E}$ extends $\Gamma$
and $s$ so that they agree. It then states that if it can reduce $t$
in the extended store to an irreducible term in $j < k$ steps, then
this term must be in a corresponding $\mathcal{V}$ set with $\Gamma$
now extended to agree with the store resulting from the reduction
steps.

$\irred {t} {s}$ is a shorthand for $\neg\exists t', s'. \reduction t s
{t'} {s'}$. $\mathcal{\supseteq}$ is used initially for the possibly
shorter store to agree with the environment, and can extend both in
many different ways. $\mathcal{\supseteq}^!$ is used finally for the
possibly shorter environment to agree with the store, and just extends
the environment in one straightforward way: hence, it defines
singleton sets.

\begin{align*}
&\rele k \Gamma s T = \{ t \ |\\
&\gap k = 0 \orl (\forall j < k,\\
&\gap\gap \forall (\Gamma'; s') \in \rels k \Gamma s ,\\
&\gap\gap \reductionl t {s'} {t'} {s''} j \andl\\
&\gap\gap \irred {t'} {s''} \impliesl\\
&\gap\gap \forall \Gamma'' \in \relg k {s''} {\Gamma'} ,\\
&\gap\gap {t'} \in \relv {k-j-1} {\Gamma''} {s''} T)\\
&\}
\end{align*}

\subsubsection{Extending the environment and the store}
$\rels k \Gamma s$ for $k > 0$ defines the set of completed
environment and stores that agree on $k-1$ steps, and that extend
$\Gamma$ and $s$. $s$ must agree with a {\it prefix} of $\Gamma$.
Both $\Gamma$ and $s$ are ordered maps. For $s$, $s'$ extends $s$ if
$s$ is a prefix of $s'$. For $\Gamma$, $\Gamma'$ extends $\Gamma$ if
we get back $\Gamma$ by keeping only the elements of $\Gamma'$ that
belong to $\Gamma$. Furthermore, a prefix of $\Gamma'$ agrees with
$s$.
\begin{align*}
&\rels k \Gamma s = \{\\
&\ (\seq{x : T}^m, \seq{x_{ij} : T_{ij}}^{m \leq i < n; 0 \leq j < i_n}; s, \seq{x_{ij} \mapsto c_{ij}}^{m \leq i < n; 0 \leq j < i_n}) |\\
&\ s = \seq{x \mapsto c}^m \andl \Gamma = \seq{x : T}^n \andl \\
&\ m\leq n \andl \forall i, m \leq i < n, \forall {i_n}, j, 0 \leq j < i_n,\\
&\ \forall T_{ij}, c_{ij}, T_{i{(i_n-1)}} = T_i, \forall n' \leq n, i_n' \leq i_n,\\
&\ c_{ij} \in \relv {k-1} {\seq{x : T}^m, \seq{x_{ij} : T_{ij}}^{m \leq i' < n; 0 \leq j < i_n'}} {s, \seq{x_{ij} \mapsto c_{ij}}^{m \leq i < n'; 0 \leq j < i_n'}} {T_{ij}}\\
&\}
\end{align*}

\subsubsection{Completing the environment to agree with the store}
$\relg k s \Gamma$ defines a singleton set of a completed environment
that agrees with a store $s$ by simply copying the constructor type
from the store for each missing variable.
\begin{align*}
&\relg k s \Gamma = \{ \Gamma, \seq{x_i : T_{c_i}}^{m \leq i < n} \ |\\
&\gap \Gamma = \seq{x : T}^m \andl s = \seq{x \mapsto c}^n\\
&\}
\end{align*}

\subsubsection{Terms in the Logical Relation}

$\Gamma \ds t : T$ is simply defined as $t \in \rele k \Gamma \emptyset T,
\forall k$.

\subsection{Statements and Proofs}

\subsubsection{Fundamental Theorem}

The fundamental theorem is the implication from $\Gamma \ts t : T$ to
$\Gamma \ds t : T$. Type safety is a straightforward corollary of this
theorem.

\begin{myproof}
The proof is on induction on the derivation of $\Gamma \ts t : T$. For
each case, we need to show $t \in \rele k \Gamma \emptyset T, \forall
k$. The non-trivial case is when $k > 0$ and for $(\Gamma'; s') \in
\rels k \Gamma s$ and some $j < k$, $\reductionl {t} s {t'} {s'} j
\andl \irred {t'} {s'}$. Then, we need to show ${t'} \in \relv {k-j-1}
      {\Gamma''} {s'} T$ for ${\Gamma''} \in \relg \Gamma k {s'}
      {\Gamma'}$.

\ \\

\Case \textsc{var}: $\Gamma \ts x : T$ knowing $(x : T) \in
\Gamma$. $x \in \relv {k-1} {\Gamma'} s T$ follows from the definition of
$\rels k \Gamma \emptyset$.

\ \\

\Case \textsc{sel}: $\Gamma \ts t_1.l_i : T$ knowing $\Gamma \ts t_1 :
T_1$, $\Gamma \ts T_1 \expand_z \seq{D}$, $\ldecl {l_i} {V_i} \in
\seq{D}$ and knowing either that $t_1 = p_1 \andl T = \subst p z
    {V_i}$ or that $z \not\in \fn(V_i) \andl T = V_i$.

By operational semantics and induction hypothesis, $\reductionl {t_1}
s {t_1'} {s'} {j-1}$ and $\irred {t_1'} {s'}$ and ${t_1'} \in \relv
{k-j+1-1} {\Gamma'} {s'} {T_1}$.

By operational semantics and the above, $\reductionl {t_1'.l_i} {s'}
{t'} {s'} 1$, and we can conclude $t' \in \relv {k-j-1} {\Gamma''} {s'}
T$ from the clause for value labels of ${t_1'} \in \relv {k-j}
{\Gamma''} {s'} {T_1}$.

\ \\

\Case \textsc{msel}: $\Gamma \ts t_1.m_i(t_2) : T$ knowing $\Gamma \ts
t_1 : T_1$, $\Gamma \ts t_2 : T_2$, $\Gamma \ts T_1 \expand_z
\seq{D}$, $\mdecl {m_i} {S_i} {U_i} \in \seq{D}$ and knowing either
that $t_1 = p_1 \andl S = \subst p z {S_i} \andl T = \subst p z {U_i}$
or that $z \not\in \fn(S_i) \andl z \not\in \fn(U_i) \andl S = S_i \andl
T = U_i$, and knowing that $\Gamma \ts T_2 \sub S$.

By operational semantics and induction hypotheses, $\reductionl {t_1}
s {t_1'} {s_1} {j_1}$ and $\irred {t_1'} {s_1}$ and $\reductionl {t_2}
{s} {t_2'} {s_2} {j_2}$ and $\irred {t_2'} {s_2}$ and ${t_1'} \in
\relv {k-j_1-1} {\Gamma_1} {s_1} {T_1}$ and ${t_2'} \in \relv
{k-j_2-1} {\Gamma_2} {s_2} {T_2}$.

Because $t_2$ reduces to a value $t_2'$ starting in store $s$, it
should also reduce to a value $v_2$ in the same number of steps
starting in store $s_1$, since $s_1$ extends $s$. So let $\reductionl
{t_2} {s_1} {v_2} {s_{12}} {j_2}$ with $v_2 \in \relv {k-j_2-1}
{\Gamma_{12}} {s_{12}} {T_2}$.

By the above and operational semantics, $\reductionl {t_1'.m_i(v_2)}
{s_{12}} {\subst {v_2} {x_i} {t_i}} {s_{12}} 1$.

By the substitution lemma, ${\subst {v_2} {x_i} {t_i}} \in \rele
{k-\max(j_1,j_2)-1} {\Gamma_{12}} {s_{12}} {T}$. Supposing,
$\reductionl {\subst {v_2} {x_i} {t_i}} {s_{12}} {t'} {s'} {j_3}$,
with $j_1 + j_2 + j_3 + 1 = j$, this completes the case, by
monotonicity of $\mathcal{V}$.

\ \\

\Case \textsc{new}: $\Gamma \ts \new y c {t_b} : T$ knowing ...

By operational semantics, $\reductionl {\new y c {t_b}} s {t_b} {s_b} 1$
where $s_b = s \envplus{y \mapsto c}$. So $\reductionl {t_b} {s_b} {t'}
{s'} {j-1}$.

By induction hypotheses, $y \in \relv k {\Gamma_b} {s_b} {T_c}$ and $t_b \in
\rele k {\Gamma_b} {s_b} {T}$.

Result follows by monotonicity of $\mathcal{V}$.

\end{myproof}

\subsubsection{Substitution Lemma}

The substitution lemma states that if (1) $v \in \relv {k_2}
{\Gamma_{12}} {s_{12}} {T_2}$ and (2) $t \in \rele {k_1} {\Gamma_1
  \envplus{{x} : S}} {s_1} T$ and (3) $\Gamma \ts {T_2} \sub S$ with
(4) $x \not\in \fn(T)$ and $\Gamma_1$ extends $\Gamma$ and
$\Gamma_{12}$ extends $\Gamma_1$ and $s_{12}$ extends $s_1$ and
$\Gamma_1$ agrees with $s_1$ and $\Gamma_{12}$ agrees with $s_{12}$
and a prefix of $\Gamma_{12}$ agrees with $s_1$, then $\subst {v} {x}
{t} \in \rele {\min(k_1,k_2)} {\Gamma_{12}} {s_{12}} T$.

\begin{proofsketch}
By (1) and (3), it should hold that (5) $v \in \relv {k_2}
{\Gamma_{12}} {s_{12}} {S}$ by the subset semantics lemma.  Since (2)
holds, it should also hold that ${t} \in \rele {\min(k_1,k_2)}
{\Gamma_{12} \envplus{{x} : S}} {s_{12}} T$ by the extended
monotonicity lemma. Then, we can instantiate $x$ in the complete store
to map to what $v$ maps to. This should be fine by (5) and
monotonicity. Thus, ${t} \in \rele {\min(k_1,k_2)} {\Gamma_{12}
  \envplus{{x} : S}} {s_{12} \envplus{{x} \mapsto s_{12}(v)}}
T$. Thanks to (4), we don't actually need $x$ to be held abstract in
the environment, because it won't occur in $T$ or its expansion (a
{\it potential pitfall} is whether its occurrences in $t_i$ could
still cause a check to fail through narrowing issues), so we can use
the type of $v$ in the environment instead of $S$ for $x$: ${t} \in
\rele {\min(k_1,k_2)} {\Gamma_{12} \envplus{{x} : {\Gamma_{12}(v)}}}
      {s_{12} \envplus{{x} \mapsto s_{12}(v)}} T$. This implies what
      needs to be shown.
\end{proofsketch}

\subsubsection{Subset Semantics Lemma}

The subset semantics lemma states that if $v \in \relv k \Gamma s S$
and $\Gamma \ts S <: U$, then $v \in \relv k \Gamma s U$.

\begin{proofsketch}
Because $S$ is a subtype of $U$, it should hold that the expansion of
$S$ subsumes the expansion of $U$, when the ``self'' occurrences are
of type $S$. Therefore, for $v \in \relv k \Gamma s U$, we have fewer
declarations to check than for $v \in \relv k \Gamma s S$.

A {\it potential pitfall} is whether some types of the expansion of
$U$ can become non-expanding when the ``self'' occurrences are
actually $v$ instead of just abstractly of type $S$, causing a check
to fail. Another worry is that such a non-expanding type results from
narrowing of a parameter type.
%However, we believe this is fine by
%subtyping regularity wtr to $\nswfe$ and $v \in \relv k \Gamma s S$
%which cheks bounds and $\nswfe$ of the type labels.
\end{proofsketch}

\subsubsection{Extended Monotonicity Lemma}

The extended monotonicity lemma states that if $t \in \rele k {\Gamma
  \envplus{x : S}} s T$ then $t \in \rele j {\Gamma' \envplus{x : S}}
{s'} T$ for $j \leq k$, $\Gamma'$ extends $\Gamma$, $s'$
extends $s$, and $\Gamma$ agrees with $s$ and a prefix of $\Gamma'$
agrees with $s$.

\begin{proofsketch}
For the monotonicity with regards to the step index, this follows
directly from the definitions of $\mathcal{E}$ and $\mathcal{V}$. For
the environment and the store, this follows by design from the
definition of $\rels k {\Gamma \envplus{x : S}} s$. To extend the
environment and the store for $x : S$, we can append as much as we
want to $\Gamma$ and $s$, to get $\Gamma'$ and $s'$, and then ignore
the last element which is for $x : S$.
\end{proofsketch}

\section{Discussion}\label{discussion}

\subsection{Why No Inheritance?}\label{why-no-inheritance}

In the calculus we made the deliberate choice not to model any form of
inheritance. This is, first and foremost, to keep the calculus simple.
Secondly, there are many different approaches to inheritance and
mixin composition, so that it looks advantageous not to tie the basic
calculus to a specific one. Finally, it seems that the modelization of
inheritance lends itself to a different approach than the basic
calculus. For the latter, we need to prove type safety of the calculus.
One might try to do this also for a calculus with inheritance, but our
experience suggests that this complicates the proofs considerably.  An
alternative approach that might work better is to model inheritance
as a form of code-reuse. Starting with an enriched type system with
inheritance, and a translation to the basic calculus, one needs to
show type safety wrt the translation. This might be easier than
to prove type safety wrt reduction.
