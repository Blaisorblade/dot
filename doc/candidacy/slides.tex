\documentclass{beamer}

\usepackage{minted}
\usemintedstyle{eclipse}

\begin{document}

\begin{frame}
\frametitle{DOT: Dependent Object Types}

The DOT calculus proposes a new \emph{type-theoretic foundation} for Scala
and languages like it. It models
\begin{itemize}
\item \emph{path-dependent types}
\item abstract type members
\item mixture of nominal and structural typing via refinement types
\end{itemize}

It does not model
\begin{itemize}
\item inheritance and mixin composition
\item what's currently in Scala
\end{itemize}

\end{frame}

\begin{frame}[fragile]{Path-dependent types}

\begin{description}[path-dependent type]
\item[path-dependent type] limited form of \emph{dependent type}, in which a type depends on a \emph{path}
\item[dependent type] a type which depends on a term
\item[path] a chain of immutable fields or variables
\end{description}

\inputminted[frame=lines,fontsize=\footnotesize]{scala}{intro.scala}

\end{frame}

\begin{frame}[fragile]{Hello}
\begin{minted}{scala}
object HelloWorld extends App {
  println("Hello, World")
}
\end{minted}
\end{frame}


\end{document}
