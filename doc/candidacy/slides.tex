\documentclass{beamer}

\usepackage{minted}
\usemintedstyle{eclipse}

\title{Dependent Object Types}
\subtitle{EDIC Candidacy Exam}
\author{Nada Amin}
\institute{LAMP, I\&C, EPFL}
\date{September 12, 2012}

\begin{document}

\frame{\titlepage}

\section{Introduction}

\begin{frame}
\frametitle{DOT: Dependent Object Types}

The DOT calculus proposes a new \emph{type-theoretic foundation} for Scala
and languages like it. It models
\begin{itemize}
\item \emph{path-dependent types}
\item abstract type members
\item mixture of nominal and structural typing via refinement types
\end{itemize}

It does not model
\begin{itemize}
\item inheritance and mixin composition
\item what's currently in Scala
\end{itemize}

\end{frame}

\begin{frame}[fragile]{Path-dependent types}

\begin{description}[path-dependent type]
\item[path-dependent type] limited form of \emph{dependent type}, in which a type depends on a \emph{path}
\item[dependent type] a type which depends on a term
\item[path] a chain of immutable fields or stable values
\end{description}

\inputminted[frame=lines,fontsize=\footnotesize]{scala}{intro.scala}
\end{frame}

\begin{frame}{Selected Papers}
\begin{enumerate}
\item {\it A Type-Theoretic Approach to Higher-Order
  Modules with Sharing}\\{\footnotesize\textcolor{gray}{by Robert Harper and Mark
  Lillibridge}}\begin{itemize}
\item calculus for modules in the ML tradition
\item modules are first-class values
\end{itemize}
\item {\it Tribe: a simple virtual class calculus}\\
  {\footnotesize\textcolor{gray}{by Dave Clarke, Sophia Drossopoulou, James Noble and Tobias Wrigstad}}\begin{itemize}
\item calculus for virtual classes
\item paths are types, and types are ``generalized'' paths
\end{itemize}
\item {\it A Nominal Theory of Objects with Dependent
  Types}\\{\footnotesize\textcolor{gray}{by Martin Odersky, Vincent Cremet, Christine R{\"o}ckl and
  Matthias Zenger}}\begin{itemize}
\item calculus unifying advanced object-oriented and module systems
\item foundation for Scala
\end{itemize}
\end{enumerate}
\end{frame}

\section{Survey of Selected Papers}

\subsection{``Translucent'' ML}

\begin{frame}{SML modules}
\begin{description}[structure]
\item[structure] a module; a collection of types, values and
  (sub)structures packaged as a program unit.
\item[signature] an interface; the type of a structure; description of
  the types, values and (sub)structures of a structure given by their
  kinds, types and interfaces.
\item[functor] a function mapping structures to structures
\item[type-sharing] enables stating that types specified in separate
  modules are actually equal
\end{description}
% example?
\end{frame}

\begin{frame}[fragile]{SML modules: Example}
TODO
\end{frame}

\begin{frame}{Limitations of SML modules}
\begin{itemize}
\item modules are not first-class; e.g. not possible to return a
  structure from an {\tt if} expression.
\item type sharing is restricted to equality between type names, not
  general type expressions; in effect, transparency / opaqueness of a
  signature cannot be fully fined-tuned
\end{itemize}
\end{frame}

\begin{frame}{ML: Translucent Sums}
\begin{itemize}
\item translucent sums are values representing modules
\item translucent sum: sequence of bindings
\item translucent sum type: sequence of declarations
\item unlike traditional records, later fields can depend on earlier ones
\item any type can be partially or fully determined by a type
\item dependent typing since translucent sums are terms that may contain type components
\end{itemize}
\end{frame}

\begin{frame}[fragile]{ML: Example}
\begin{minted}{sml}
structure XInt = struct
  type T = int
  val v = 3
  val f = negate
end
structure XBool = struct
  type T = bool
  val v = true
  val f = not
end
structure X =
  if flip() then XInt else XBool
signature XType = sig
  type T
  val v = T
  val f = T -> T
end
\end{minted}
\end{frame}

% Maintaing Soundness?

\begin{frame}{ML: Path-Dependent Types}
\begin{itemize}
\item Scala's abstract types  $\approx$ ML's abstract types of signatures; limitations:
\begin{itemize}
\item recursive references
\item bounded quantifications
\end{itemize}
\item translucent sum can be given a more precise type by referring to the name of its member path with a path selection; essential for
\begin{itemize}
\item breaking the dependencies between (sub)fields
\item propagation of typing information
\end{itemize}
\end{itemize}
\end{frame}

\subsection{Tribe Calculus}

\begin{frame}{Tribe Calculus: Virtual Classes}
\begin{itemize}
\item {\it familes} of classes inherited together
\item classes are lexically nested inside other classes: when a class
  is inherited, its nested inner classes are inherited along with
  their methods and fields
\item two form of inheritance: subclassing and further binding
\item {\it family polymorphism}: code written for one family
  also works for extensions of that family
\item in Tribe:\begin{itemize}
\item path types can depend simultaneously on both classes and objects
\item paths can use an {\tt out} field to move from an object to the
  object which surrounds it; enables ubiquitous access to an object's
  family without the need to drag around family arguments.
\end{itemize}
\end{itemize}
\end{frame}

\begin{frame}[fragile]{Tribe: Example}
\begin{minted}{java}
class Graph {
  class Node {
    Edge connect(Node other) {
      return new Edge(this, other);
    }
  }
  class Edge {
    Node from, to;
    Edge(Node f, Node t) { from = f; to = t; }
  }
}

class ColouredGraph extends Graph { // subclassing
  class Node {                      // further binding
    Colour nodeColour;
  }
}
\end{minted}
\end{frame}

\begin{frame}[fragile]{Tribe: Example (continued)}
\begin{minted}{java}
final ColouredGraph cg1, cg2;
cg1.Node cn1, cn3;
cg2.Node cn2;
cn1.connect(cn3); // Type Correct
cn2.connect(cn3); // Type Error!!!

class Library {
  int distance(Graph.Node n1, n1.out.Node n2) { ... }

  e.out.Edge copyEdge(Graph.Edge e) {
    e.out.Node from = e.from;
    e.out.Node to = e.to;
    new e.out.Edge(from, to);
  }
}
\end{minted}
\end{frame}

\begin{frame}[fragile]{Tribe Types}
\begin{itemize}
\item In Tribe, types are a form of generalized paths: a final
  variable, {\tt this} or a class name then followed by a possibly
  empty sequence of field, {\tt out} or class selections.
\item {\tt kitt.Passenger.name}: the name of one of Kitt's passengers
\item {\tt kitt.driver <: Car.driver <: Car.Passenger <: Car.Traveller <:
  Vehicle.Traveller}.
\item {\tt kitt.driver $\not<:$ karr.driver} and {\tt kitt.Passenger $\not<:$ karr.Passenger}.
\end{itemize}
\begin{minted}[frame=lines,fontsize=\footnotesize]{java}
class Vehicle { class Traveller { ... } }
class Car extends Vehicle {
  class Passenger extends Traveller {
    class String { ... }
    final String name;
  }
  final Passenger driver;
}
final Car kitt;
final Car karr;
\end{minted}
\end{frame}

\begin{frame}{Tribe: Discussion}
\begin{itemize}
\item no virtual classes in Scala; its virtual types can emulate some (but not all) of the benefits
\item because of {\tt out} field, Tribe has limited cross-family inheritance
\item soundness? substitution lemma seems wrong when {\tt null} is
  substituted for a variable in an expression whose type contains
  that variable
\item aims to have decidable type-checking, but still not proved so
\end{itemize}
\end{frame}

% Virtual Classes
% Example
% Tribe Types (generalized paths & subtyping relation)
% Follow-up
  % Ownership System?
  % Soundness?
  % Cross-Family Inheritance

\subsection{${\nu}Obj$}

\section{Proposal}

\section{Backup Slides}

% System F omega
% System F sub
\end{document}
