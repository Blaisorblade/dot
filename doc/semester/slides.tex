\documentclass{beamer}

%%%%%%%%%%%%%%%%%%%%%%%%%%%%%%%%%%%%%%%
%   Language abstraction commands     %
%%%%%%%%%%%%%%%%%%%%%%%%%%%%%%%%%%%%%%%

%% Relations
% Subtype 
\newcommand{\sub}{<:}
% Type assignment
\newcommand{\typ}{:}
% reduction
\newcommand{\reduces}{\;\rightarrow\;}
% well-formedness
\newcommand{\wf}{\;\mbox{\textbf{wf}}}
\newcommand{\nswf}{\mbox{\textbf{wf}}}
\newcommand{\wfe}{\;\mbox{\textbf{wfe}}}
\newcommand{\nswfe}{\mbox{\textbf{wfe}}}

%% Operators
% Type selection
\newcommand{\tsel}{\#}
% Function type
\newcommand{\tfun}{\rightarrow}
\newcommand{\dfun}[3]{(#1\!:\!#2) \Rightarrow #3}
% Conjunction
\newcommand{\tand}{\wedge}
% Disjunction
\newcommand{\tor}{\vee}
% Singleton type suffix
\newcommand{\sing}{.\textbf{type}}

%% Syntax
% Header for typing rules
\newcommand{\judgement}[2]{{\bf #1} \hfill #2}
% Widening
\newcommand{\wid}[2]{#1 : #2}
% Refinement
\newcommand{\refine}[2]{\left\{#1 \Rightarrow #2 \right\}}
\newcommand{\mlrefine}[2]{\{#1 \Rightarrow #2 \}}
% Field definitions
\newcommand{\ldefs}[1]{\left\{#1\right\}}
\newcommand{\mlldefs}[1]{\{#1\}}
% Member sequences
\newcommand{\seq}[1]{\overline{#1}}
% Lambda
\newcommand{\dabs}[3]{(#1\!:\!#2)\Rightarrow #3}
\newcommand{\abs}[3]{\lambda #1\!:\!#2.#3}
% Method Application
\newcommand{\mapp}[3]{#1.#2(#3)}
% Substitution
\newcommand{\subst}[3]{[#1/#2]#3}
% Object creation
\newcommand{\new}[3]{\textbf{val }#1 = \textbf{new }#2 ;\; #3}
\newcommand{\mlnew}[3]{\textbf{val }#1 = \textbf{new }#2 ;\;\\&#3}
%\renewcommand{\new}[3]{#1 \leftarrow #2 \,\textbf{in}\, #3}
% Field declaration
\newcommand{\Ldecl}[3]{#1 : #2..#3}%{#1 \operatorname{>:} #2 \operatorname{<:} #3}
\newcommand{\ldecl}[2]{#1 : #2}
\newcommand{\mdecl}[3]{#1 : #2 \tfun #3}
% Top and Bottom
\newcommand{\Top}{\top}%{\textbf{Top}}
\newcommand{\Bot}{\bot}%\textbf{Bot}}
% Environment extension
%\newcommand{\envplus}[1]{\uplus \{ #1 \}}
\newcommand{\envplus}[1]{, #1}
% Reduction
\newcommand{\reduction}[4]{#1 \operatorname{|} #2 \reduces #3 \operatorname{|} #4}

% Sugar
\newcommand{\arrow}[2]{#1\rightarrow_s#2}
\newcommand{\fun}[4]{\textbf{fun } (#1:#2)\;#3\;#4}
\newcommand{\app}[2]{(\textbf{app }#1\;#2)}
\newcommand{\mlapp}[2]{(\textbf{app }#1\;\\&#2)}
\newcommand{\cast}[2]{(\textbf{cast }#1\;#2)}

\newcommand{\lindent}{\hspace{-4mm}}

% Logical relations
\newcommand{\relv}[4]{\mathcal{V}_{#1;#2;#3}\llbracket#4\rrbracket}
\newcommand{\rele}[4]{\mathcal{E}_{#1;#2;#3}\llbracket#4\rrbracket}
\newcommand{\rels}[3]{\mathcal{\supseteq}_{#1}\llbracket#2;#3\rrbracket}
\newcommand{\relg}[3]{\mathcal{\supseteq^!}_{#1;#2}\llbracket#3\rrbracket}
\newcommand{\irred}[2]{\text{irred }(#1,#2)}
\newcommand{\andl}{\;\wedge\;}
\newcommand{\orl}{\vee}
\newcommand{\impliesl}{\rightarrow}
\newcommand{\reductionl}[5]{#1 \operatorname{|} #2 \;\rightarrow^{#5}\; #3 \operatorname{|} #4}
\newcommand{\ds}{\,\vDash\,}


%\usepackage{beamerthemesplit}
\usecolortheme{seagull}
\useinnertheme{circles}
\useoutertheme{infolines}

\title{DOT: Dependent Object Types}
\subtitle{Semester Project, Spring 2012}
\author{Nada Amin}
\institute{EPFL}
\date{}

\begin{document}

\frame{\titlepage}

\section{Introduction}

\subsection{What is DOT?}

\begin{frame}
\frametitle{DOT: Dependent Object Types}
\begin{itemize}
\item type-theoretic foundation of Scala and languages like it
\item models:
\begin{itemize}
\item path-dependent types
\item abstract type members
\item mixture of nominal and structural typing via refinement types
\end{itemize}
\item does not model:
\begin{itemize}
\item inheritance and mixin composition
\item what's currently in Scala
\end{itemize}
\end{itemize}
\end{frame}

\begin{frame}
\frametitle{DOT Syntax}
\begin{columns}
\begin{column}[t]{5cm}
\begin{block}{term $t$}
\begin{itemize}
\item variable\\$x$
\item lambda abstraction\\$\abs x T t$
\item function application\\$\app t {t'}$
\item field selection\\$t.l$
\item object creation expression\\$\new x {T_c \ldefs{\seq{l = v}}} t$
\end{itemize}
\end{block}
\end{column}
\begin{column}[t]{5cm}
\begin{block}{type $T$}
\begin{itemize}
\item selection\\$p.L$
\item refinement\\$T \refine z {\seq{D}}$
\item function\\$T \tfun T'$
\item intersection\\$T \tand T'$
\item union\\$T \tor T'$
\item $\Top$, $\Bot$
\end{itemize}
\end{block}
\end{column}
\end{columns}
\end{frame}

\begin{frame}
\frametitle{DOT Judgments}
\begin{columns}
\begin{column}[t]{5cm}
\begin{block}{Typing Judgments}
\begin{itemize}
\item type assignment\\$\Gamma \ts t \typ T$
\item subtyping\\$\Gamma \ts S \sub T$
\item well-formedness\\$\Gamma \ts T \wf$
\item membership\\$\Gamma \ts t \ni D$
\item expansion\\$\Gamma \ts T \expand_z \seq{D}$
\end{itemize}
\end{block}
\end{column}
\begin{column}[t]{5cm}
\begin{block}{Small-Step Operational Semantics}
\begin{itemize}
\item reduction\\$\reduction t s {t'} {s'}$
\end{itemize}
\end{block}
\end{column}
\end{columns}
\end{frame}

\subsection{DOT Program Example}

\begin{frame}
\frametitle{Basics}
\framesubtitle{Booleans, Error, \ldots}
\begin{align*}
&\mlnew {\mi{root}} {\Top \mlrefine r {\\
&\ \Ldecl {\mi{Unit}} \Bot \Top\\
&\ \ldecl {\mi{unit}} {\Top \tfun {r.\mi{Unit}}}\\
&\ \Ldecl {\mi{Boolean}} \Bot {\Top \mlrefine z {\\
&\ \gap \ldecl {\mi{ifNat}} {(r.\mi{Unit} \tfun r.\mi{Nat}) \tfun (r.\mi{Unit} \tfun r.\mi{Nat}) \tfun r.\mi{Nat}}\\
&\gap}}\\
&\ \ldecl {\mi{false}} {{r.\mi{Unit}} \tfun {r.\mi{Boolean}}}\\
&\ \ldecl {\mi{true}} {{r.\mi{Unit}} \tfun {r.\mi{Boolean}}}\\
&\ \ldecl {\mi{error}} {{r.\mi{Unit}} \tfun \Bot}\\
&\ \ldots\\
&}{\ldefs{\ldots \seq{(l = v)} \ldots }}}{ \ldots }
\end{align*}
\end{frame}

\begin{frame}
\frametitle{Basics (Continued)}
\framesubtitle{Booleans, Error, \ldots}
\begin{align*}
&\mlldefs{\\
&\ \mi{unit}  = \abs x \Top {\new u {\mi{root}.\mi{Unit}} u}\\
&\ \mi{false} = \abs u {\mi{root}.\mi{Unit}} {\\
&\ \gap \mlnew {\mi{ff}} {\mi{root}.\mi{Boolean} \mlldefs{\\
&\ \gap\gap {\mi{ifNat}} = \abs t {\mi{root}.\mi{Unit} \tfun \mi{root}.\mi{Nat}} {\\&\gap\gap\gap\abs e {\mi{root}.\mi{Unit} \tfun \mi{root}.\mi{Nat}}\\&\gap\gap\gap\app e {\mi{root}.\mi{unit}}}\\
&\ \gap}}{
\ \gap \mi{ff}}
&}\\
&\ \mi{error} = \abs u {\mi{root}.\mi{Unit}} {\app {\mi{root}.\mi{error}} u}\\
&\ \ldots\\
&}
\end{align*}
\end{frame}

\subsection{Contributions}
\begin{frame}
\frametitle{Outline}
\tableofcontents
\end{frame}

\section{Counterexamples}
\frame{\tableofcontents[currentsection]}

\subsection{Subtyping Transitivity}

\begin{frame}
\frametitle{No Subtyping Transitivity to No Preservation}
\begin{columns}
\begin{column}[t]{6cm}
\begin{enumerate}
\item Start with 3 types $S$, $T$, $U$ st $S \sub T$ and $T \sub U$ but $S \not\sub U$.
\item Create function of type $S \tfun S$.
\item Cast it to $S \tfun T$.
\item Cast it to $S \tfun U$.
\item After some reduction step, the first cast vanishes and we need to cast directly from $S \tfun S$ to $S \tfun U$.
\end{enumerate}
Note: The 3 types don't need to be realizable but must be expressible within a realizable universe.
\end{column}
\begin{column}[t]{4cm}
\begin{block}{Code Recipe}
\begin{align*}
&\mlnew u \ldots {
\ \mlapp{\abs x \Top x}{
\ \ \mlapp{\abs f {S \tfun U} f}{
\ \ \ \mlapp{\abs f {S \tfun T} f}{
\ \ \ \ \mlapp{\abs f {S \tfun S} f}{
\ \ \  \ \ \abs x S x}}}}}
\end{align*}
\end{block}
\end{column}
\end{columns}
\end{frame}

\begin{frame}
\frametitle{Non-Expanding Types and Subtyping Transitivity}
\begin{columns}
\begin{column}[m]{5cm}
\begin{align*}
\Top & \mlrefine u {\\
&\ \Ldecl {\mathit{Bad}} {\Bot} {u.\mathit{Bad}}\\
&\ \Ldecl {\mathit{Good}} {\Top \refine z {\Ldecl L \Bot \Top}} {\\&\gap\gap\Top \refine z {\Ldecl L \Bot \Top}}\\
&\ \Ldecl {\mathit{Lower}} {u.\mathit{Bad} \tand u.\mathit{Good}} {u.\mathit{Good}}\\
&\ \Ldecl {\mathit{Upper}} {u.\mathit{Good}} {u.\mathit{Bad} \tor u.\mathit{Good}}\\
&\ \Ldecl X {u.\mathit{Lower}} {u.\mathit{Upper}}\\
&}\\
\\
S &= u.\mathit{Bad} \tand u.\mathit{Good}\\
T &= u.\mathit{Lower}\\
U &= u.X \refine z {\Ldecl L \Bot \Top}
\end{align*}
\end{column}
\begin{column}[m]{5cm}
\begin{block}{\textsc{tsel-$\expand$}}
  \infrule
  {\Gamma \ts p \ni \Ldecl L S U \spcomma U \expand_z \seq D}
  {\Gamma \ts p.L \expand_z \seq D}
\end{block}
\begin{block}{\textsc{$\sub$-rfn}}
  \infrule
   {\Gamma \ts S \sub T \spcomma S \expand_z \seq{D'} \\
   \Gamma \envplus{z: S} \ts \seq{D' \sub D}}
   {\Gamma \ts S \sub T \refine z {\seq D}}
\end{block}
\end{column}
\end{columns}
\end{frame}

\subsection{Narrowing}

\begin{frame}
\frametitle{Functions as Objects}
\center{// $f$ is an undefined function!}
\begin{align*}
&\mlnew u {\Top \refine z {\Ldecl C {\Top \tfun \Top} {\Top \tfun \Top}} \ldefs{}} {
\mlnew f {u.C \ldefs{}} {
\app {(\abs g {\Top \tfun \Top} {\app g (\abs x \Top x)})} f
}}
\end{align*}
\begin{block}{\textsc{app}}
  \infrule
  {\Gamma \ts t \typ  S \tfun T \spcomma t' \typ T' \spcomma T' \sub S}
  {\Gamma \ts \app t {t'} \typ T}
\end{block}
\end{frame}

\begin{frame}
\frametitle{(Expansion and) Well-Formedness Lost}
\center{// $y.A$ is not well-formed after $v$ is substituted for $x$.}
\begin{align*}
&\mlnew v {\Top \refine z {\Ldecl L \Bot {\Top \refine z {\Ldecl A \Bot \Top, \Ldecl B {z.A} {z.A}}}} \ldefs{}}{
\mlapp {\abs x {\Top \refine z {\Ldecl L \Bot {\Top \refine z {\Ldecl A \Bot \Top, \Ldecl B \Bot \Top}}}} {\\&\gap
\mlnew z {\Top \refine z {\ldecl l {\Bot \tfun \Top}}\mlldefs{\\&\gap\gap l = \abs y {x.L \tand {\Top \refine z {\Ldecl A {z.B} {z.B}, \Ldecl B \Bot \Top}}} {\\&\gap\gap\gap\abs a {y.A} {\app {(\abs x \Top x)} a}}}}{
\ \app {(\abs x \Top x)} z
}}}{\ v}}
\end{align*}
\end{frame}

\begin{frame}
\frametitle{\textsc{term-$\ni$} Restriction}
\begin{columns}
\begin{column}[m]{5cm}
\begin{align*}
&X = \Top \mlrefine z {
\Ldecl {L_a} \Top \Top,
\ldecl l {z.L_a}
}\\
&Y = \Top \mlrefine z {
\ldecl l \Top
}\\
&\mlnew u {X \ldefs{ l = u }}{
\mlapp {\abs y {\Top \tfun Y} {\app y u}} {\ (\abs d \Top {\app {(\abs x X x)} u})}.l
}
\end{align*}
\end{column}
\begin{column}[m]{5cm}
\begin{block}{\textsc{path-$\ni$}}
  \infrule
  {\Gamma \ts p \typ T \spcomma T \expand_z \seq D}
  {\Gamma \ts p \ni \subst p z {D_i}}
\end{block}
\begin{block}{\textsc{term-$\ni$}}
  \infrule
  {z \not\in \fn(D_i)\\ \andalso \Gamma \ts t \typ T \spcomma T \expand_z \seq D}
  {\Gamma \ts t \ni D_i}
\end{block}
\end{column}
\end{columns}
\end{frame}

\subsection{Path Equality}
\begin{frame}
\frametitle{Path Equality}
\center{// $a.i.l$ reduces to $b.l$. $b.l$ has type $b.X$, so we need $b.X <: a.i.X$.}
\begin{align*}
&\mlnew b {\Top \mlrefine z {&&\Ldecl X \Top \Top&\\
&&&\ \ldecl l {z.X}&}\ldefs{l = b}}{
\mlnew a {\Top \mlrefine z {\ldecl i {\Top \mlrefine z {&&&\\
&&&\Ldecl X \Bot \Top\\
&&&\ldecl l {z.X}}}&}\ldefs{i = b}}}{
\app {(\abs x \Top x)} {(\app {(\abs x {a.i.X} x)} {a.i.l})}}
\end{align*}
\end{frame}

\section{Patches}
\frame{\tableofcontents[currentsection]}

\begin{frame}
\frametitle{Non-Expanding Types and Subtyping Transitivity}
\begin{enumerate}
\item Introduce a new judgment form for whether a type is well-formed and expanding.
\begin{block}{$\Gamma \ts T \wfe$}
  \infrule
  {\Gamma \ts T \wf \spcomma T \expand_z \seq{D}}
  {\Gamma \ts T \wfe}
\end{block}
\item Replace other uses of $\wf$ with $\wfe$.
\item Make subtyping regular with respect to $\wfe$.
\end{enumerate}
\end{frame}

\begin{frame}
\frametitle{Functions as Sugar}
\framesubtitle{like in Scala}
\begin{enumerate}
\item Introduce a new kind of label for methods with one parameter: $m : S \tfun U$.
\item Return types are now explicit.
\end{enumerate}
\end{frame}

\begin{frame}
\frametitle{Explicit Widening}
\begin{enumerate}
\item Add widening terms $\wid t T$.
\item Add widening values $\wid v T$.
\item Replace relaxed subtyping in \textsc{app} and \textsc{new} with
  ``equality''.
\item Add the only typing rule with relaxed subtyping.
\begin{block}{\textsc{wid}}
  \infrule
  {\Gamma \ts t \typ T' \spcomma T' \sub T}
  {\Gamma \ts (\wid t T) \typ T}
\end{block}
\item Reduction becomes complicated and dependent on typing judgments
  because type widenings need to be propagated outwards.
\end{enumerate}
\end{frame}

\begin{frame}
\frametitle{Path Equality}
\begin{enumerate}
\item Add store to context of typing judgments.
\item Add path-equality provisions in subtyping.
\begin{block}{\textsc{$\sub$-path}}
  \infrule
  {\Gamma, s \ts p \reduces q \spcomma T \sub q.L}
  {\Gamma, s \ts T \sub p.L}
\end{block}
\begin{block}{\textsc{path-$\sub$}}
  \infrule
  {\Gamma, s \ts p \reduces q \spcomma q.L \sub T}
  {\Gamma, s \ts p.L \sub T}
\end{block}
\item Path reduction not quite like reduction, since it roughly skips
  over widenings.
\end{enumerate}
\end{frame}

\begin{frame}
\frametitle{The Final Blow}
\framesubtitle{by the much needed $\textsc{path-$\sub$}$}
\center{//$d$ has type $\Bot$ if ignoring the cast on $b$}
\begin{align*}
&\mlnew a {\Top \refine z {\Ldecl C \Bot {\Top \refine z {\Ldecl D \Bot {z.X}, \Ldecl X \Bot \Top}}}}{
\mlnew b {{a.C} \refine z {\Ldecl X \Bot \Bot}}}{
\mlnew c {a.C}}{
\mlnew d {(\wid b {a.C}).D}}{
\app {(\abs x \Bot {x.\mathit{foo}})} d
}
\end{align*}
\end{frame}

\section{Conclusion}
\begin{frame}
\frametitle{Conclusion}
\begin{enumerate}
\item No soundness proof but lots of soundness holes.
\item Patches to calculus affect its elegance.
\item Going forward by getting broader perspective with PL summer
  school and candidacy exam on dependent object types.
\item Will keep DOT in mind, and come back to it with fresh and broader perspective.
\end{enumerate}
\end{frame}

\end{document}
