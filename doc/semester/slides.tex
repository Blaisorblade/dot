\documentclass{beamer}

%%%%%%%%%%%%%%%%%%%%%%%%%%%%%%%%%%%%%%%
%   Language abstraction commands     %
%%%%%%%%%%%%%%%%%%%%%%%%%%%%%%%%%%%%%%%

%% Relations
% Subtype 
\newcommand{\sub}{<:}
% Type assignment
\newcommand{\typ}{:}
% reduction
\newcommand{\reduces}{\;\rightarrow\;}
% well-formedness
\newcommand{\wf}{\;\mbox{\textbf{wf}}}
\newcommand{\nswf}{\mbox{\textbf{wf}}}
\newcommand{\wfe}{\;\mbox{\textbf{wfe}}}
\newcommand{\nswfe}{\mbox{\textbf{wfe}}}

%% Operators
% Type selection
\newcommand{\tsel}{\#}
% Function type
\newcommand{\tfun}{\rightarrow}
\newcommand{\dfun}[3]{(#1\!:\!#2) \Rightarrow #3}
% Conjunction
\newcommand{\tand}{\wedge}
% Disjunction
\newcommand{\tor}{\vee}
% Singleton type suffix
\newcommand{\sing}{.\textbf{type}}

%% Syntax
% Header for typing rules
\newcommand{\judgement}[2]{{\bf #1} \hfill #2}
% Widening
\newcommand{\wid}[2]{#1 : #2}
% Refinement
\newcommand{\refine}[2]{\left\{#1 \Rightarrow #2 \right\}}
\newcommand{\mlrefine}[2]{\{#1 \Rightarrow #2 \}}
% Field definitions
\newcommand{\ldefs}[1]{\left\{#1\right\}}
\newcommand{\mlldefs}[1]{\{#1\}}
% Member sequences
\newcommand{\seq}[1]{\overline{#1}}
% Lambda
\newcommand{\dabs}[3]{(#1\!:\!#2)\Rightarrow #3}
\newcommand{\abs}[3]{\lambda #1\!:\!#2.#3}
% Method Application
\newcommand{\mapp}[3]{#1.#2(#3)}
% Substitution
\newcommand{\subst}[3]{[#1/#2]#3}
% Object creation
\newcommand{\new}[3]{\textbf{val }#1 = \textbf{new }#2 ;\; #3}
\newcommand{\mlnew}[3]{\textbf{val }#1 = \textbf{new }#2 ;\;\\&#3}
%\renewcommand{\new}[3]{#1 \leftarrow #2 \,\textbf{in}\, #3}
% Field declaration
\newcommand{\Ldecl}[3]{#1 : #2..#3}%{#1 \operatorname{>:} #2 \operatorname{<:} #3}
\newcommand{\ldecl}[2]{#1 : #2}
\newcommand{\mdecl}[3]{#1 : #2 \tfun #3}
% Top and Bottom
\newcommand{\Top}{\top}%{\textbf{Top}}
\newcommand{\Bot}{\bot}%\textbf{Bot}}
% Environment extension
%\newcommand{\envplus}[1]{\uplus \{ #1 \}}
\newcommand{\envplus}[1]{, #1}
% Reduction
\newcommand{\reduction}[4]{#1 \operatorname{|} #2 \reduces #3 \operatorname{|} #4}

% Sugar
\newcommand{\arrow}[2]{#1\rightarrow_s#2}
\newcommand{\fun}[4]{\textbf{fun } (#1:#2)\;#3\;#4}
\newcommand{\app}[2]{(\textbf{app }#1\;#2)}
\newcommand{\mlapp}[2]{(\textbf{app }#1\;\\&#2)}
\newcommand{\cast}[2]{(\textbf{cast }#1\;#2)}

\newcommand{\lindent}{\hspace{-4mm}}

% Logical relations
\newcommand{\relv}[4]{\mathcal{V}_{#1;#2;#3}\llbracket#4\rrbracket}
\newcommand{\rele}[4]{\mathcal{E}_{#1;#2;#3}\llbracket#4\rrbracket}
\newcommand{\rels}[3]{\mathcal{\supseteq}_{#1}\llbracket#2;#3\rrbracket}
\newcommand{\relg}[3]{\mathcal{\supseteq^!}_{#1;#2}\llbracket#3\rrbracket}
\newcommand{\irred}[2]{\text{irred }(#1,#2)}
\newcommand{\andl}{\;\wedge\;}
\newcommand{\orl}{\vee}
\newcommand{\impliesl}{\rightarrow}
\newcommand{\reductionl}[5]{#1 \operatorname{|} #2 \;\rightarrow^{#5}\; #3 \operatorname{|} #4}
\newcommand{\ds}{\,\vDash\,}


%\usepackage{beamerthemesplit}
\usecolortheme{seagull}
\useinnertheme{circles}
\useoutertheme{infolines}

\title{DOT: Dependent Object Types}
\subtitle{Semester Project, Spring 2012}
\author{Nada Amin}
\institute{EPFL}
\date{}

\begin{document}

\frame{\titlepage}

\section{Introduction}

\subsection{What is DOT?}

\begin{frame}
\frametitle{DOT: Dependent Object Types}
\begin{itemize}
\item type-theoretic foundation of Scala and languages like it
\item models:
\begin{itemize}
\item path-dependent types
\item abstract type members
\item mixture of nominal and structural typing via refinement types
\end{itemize}
\item does not model:
\begin{itemize}
\item inheritance and mixin composition
\item what's currently in Scala
\end{itemize}
\end{itemize}
\end{frame}

\begin{frame}
\frametitle{DOT Syntax}
\begin{columns}
\begin{column}[t]{5cm}
\begin{block}{term $t$}
\begin{itemize}
\item variable\\$x$
\item lambda abstraction\\$\abs x T t$
\item function application\\$\app t {t'}$
\item field selection\\$t.l$
\item object creation expression\\$\new x {T_c \ldefs{\seq{l = v}}} t$
\end{itemize}
\end{block}
\end{column}
\begin{column}[t]{5cm}
\begin{block}{type $T$}
\begin{itemize}
\item selection\\$p.L$
\item refinement\\$T \refine z {\seq{D}}$
\item function\\$T \tfun T'$
\item intersection\\$T \tand T'$
\item union\\$T \tor T'$
\item $\Top$, $\Bot$
\end{itemize}
\end{block}
\end{column}
\end{columns}
\end{frame}

\begin{frame}
\frametitle{DOT Judgments}
\begin{columns}
\begin{column}[t]{5cm}
\begin{block}{Typing Judgments}
\begin{itemize}
\item type assignment\\$\Gamma \ts t \typ T$
\item subtyping\\$\Gamma \ts S \sub T$
\item well-formedness\\$\Gamma \ts T \wf$
\item membership\\$\Gamma \ts t \ni D$
\item expansion\\$\Gamma \ts T \expand_z \seq{D}$
\end{itemize}
\end{block}
\end{column}
\begin{column}[t]{5cm}
\begin{block}{Small-Step Operational Semantics}
\begin{itemize}
\item reduction\\$\reduction t s {t'} {s'}$
\end{itemize}
\end{block}
\end{column}
\end{columns}
\end{frame}

\subsection{DOT Program Example}

\begin{frame}
\frametitle{Basics}
\framesubtitle{Booleans, Error, \ldots}
\begin{align*}
&\mlnew {\mi{root}} {\Top \mlrefine r {\\
&\ \Ldecl {\mi{Unit}} \Bot \Top\\
&\ \ldecl {\mi{unit}} {\Top \tfun {r.\mi{Unit}}}\\
&\ \Ldecl {\mi{Boolean}} \Bot {\Top \mlrefine z {\\
&\ \gap \ldecl {\mi{ifNat}} {(r.\mi{Unit} \tfun r.\mi{Nat}) \tfun (r.\mi{Unit} \tfun r.\mi{Nat}) \tfun r.\mi{Nat}}\\
&\gap}}\\
&\ \ldecl {\mi{false}} {{r.\mi{Unit}} \tfun {r.\mi{Boolean}}}\\
&\ \ldecl {\mi{true}} {{r.\mi{Unit}} \tfun {r.\mi{Boolean}}}\\
&\ \ldecl {\mi{error}} {{r.\mi{Unit}} \tfun \Bot}\\
&\ \ldots\\
&}{\ldefs{\ldots \seq{(l = v)} \ldots }}}{ \ldots }
\end{align*}
\end{frame}

\begin{frame}
\frametitle{Basics (Continued)}
\framesubtitle{Booleans, Error, \ldots}
\begin{align*}
&\mlldefs{\\
&\ \mi{unit}  = \abs x \Top {\new u {\mi{root}.\mi{Unit}} u}\\
&\ \mi{false} = \abs u {\mi{root}.\mi{Unit}} {\\
&\ \gap \mlnew {\mi{ff}} {\mi{root}.\mi{Boolean} \mlldefs{\\
&\ \gap\gap {\mi{ifNat}} = \abs t {\mi{root}.\mi{Unit} \tfun \mi{root}.\mi{Nat}} {\\&\gap\gap\gap\abs e {\mi{root}.\mi{Unit} \tfun \mi{root}.\mi{Nat}}\\&\gap\gap\gap\app e {\mi{root}.\mi{unit}}}\\
&\ \gap}}{
\ \gap \mi{ff}}
&}\\
&\ \mi{error} = \abs u {\mi{root}.\mi{Unit}} {\app {\mi{root}.\mi{error}} u}\\
&\ \ldots\\
&}
\end{align*}
\end{frame}

\subsection{Contributions}
\begin{frame}
\frametitle{Outline}
\tableofcontents
\end{frame}

\section{Counterexamples}

\subsection{Subtyping Transitivity}

  \begin{frame}
    \frametitle{No Subtyping Transitivity to No Preservation}
\begin{columns}
\begin{column}[t]{6cm}
\begin{enumerate}
\item Start with 3 types $S$, $T$, $U$ st $S \sub T$ and $T \sub U$ but $S \not\sub U$.
\item Create function of type $S \tfun S$.
\item Cast it to $S \tfun T$.
\item Cast it to $S \tfun U$.
\item After some reduction step, the first cast vanishes and we need to cast directly from $S \tfun S$ to $S \tfun U$.
\end{enumerate}
Note: The 3 types don't need to be realizable but must be expressible within a realizable universe.
\end{column}
\begin{column}[t]{4cm}
\begin{block}{Code Recipe}
\begin{align*}
&\mlnew u \ldots {
\ \mlapp{\abs x \Top x}{
\ \ \mlapp{\abs f {S \tfun U} f}{
\ \ \ \mlapp{\abs f {S \tfun T} f}{
\ \ \ \ \mlapp{\abs f {S \tfun S} f}{
\ \ \  \ \ \abs x S x}}}}}
\end{align*}
\end{block}
\end{column}
\end{columns}
  \end{frame}

  \begin{frame}
    \frametitle{Non-Expanding Types and Subtyping Transitivity}
\begin{align*}
\Top & \mlrefine u {\\
&\ \Ldecl {\mathit{Bad}} {\Bot} {u.\mathit{Bad}}\\
&\ \Ldecl {\mathit{Good}} {\Top \refine z {\Ldecl L \Bot \Top}} {\Top \refine z {\Ldecl L \Bot \Top}}\\
&\ \Ldecl {\mathit{Lower}} {u.\mathit{Bad} \tand u.\mathit{Good}} {u.\mathit{Good}}\\
&\ \Ldecl {\mathit{Upper}} {u.\mathit{Good}} {u.\mathit{Bad} \tor u.\mathit{Good}}\\
&\ \Ldecl X {u.\mathit{Lower}} {u.\mathit{Upper}}\\
}&
\end{align*}

\begin{align*}
S &= u.\mathit{Bad} \tand u.\mathit{Good}\\
T &= u.\mathit{Lower}\\
U &= u.X \refine z {\Ldecl L \Bot \Top}
\end{align*}
  \end{frame}

\subsection{Narrowing}

  \begin{frame}
    \frametitle{Functions as Objects}
\begin{align*}
&\mlnew u {\Top \refine z {\Ldecl C {\Top \tfun \Top} {\Top \tfun \Top}} \ldefs{}} {
\mlnew f {u.C \ldefs{}} {
\ldots
}}
\end{align*}
  \end{frame}

\subsection{Path Equality}

  \begin{frame}
    \frametitle{Path Equality}
  \end{frame}

\section{Patches}

  \begin{frame}
    \frametitle{Patches}
  \end{frame}

\section{Conclusion}

  \begin{frame}
    \frametitle{Conclusion}
  \end{frame}

\end{document}
