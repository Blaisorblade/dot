\documentclass[9pt]{sigplanconf}

% The following \documentclass options may be useful:
%
% 10pt          To set in 10-point type instead of 9-point.
% 11pt          To set in 11-point type instead of 9-point.
% authoryear    To obtain author/year citation style instead of numeric.

\usepackage{amsmath}
\usepackage{amssymb}

\usepackage{fleqn}
\usepackage{listings}
\usepackage{math}
\usepackage{amsmath}
\usepackage{latexsym}
\usepackage{bcprules}
\usepackage[scaled=0.848971]{luximono} % This is for 11 pt Default font
\usepackage[T1]{fontenc}

% Prooftree formatting
\usepackage{prooftree}

\usepackage{multicol}
\usepackage{framed}

\usepackage{stmaryrd}

%\usepackage{float}
%\floatstyle{boxed} 
%\restylefloat{figure}

% support for generating PDF files
%\newif\ifpdf
%    \ifx\pdfoutput\undefined
%    \pdffalse
%\else
%    \pdftrue
%    \pdfoutput=1
%\fi

%versions
% Use dependent function types
\newif\ifdep\depfalse

\lstset{
  literate=
  {=>}{$\Rightarrow\;$}{2}
  {<:}{$<:\;$}{1}
}

\lstdefinelanguage{scala}{% 
       morekeywords={% 
                try, catch, throw, private, public, protected, import, package, implicit, final, package, trait, type, class, val, def, var, if, for, this, else, extends, with, while, new, abstract, object, case, match, sealed,override},
         sensitive=t, % 
   morecomment=[s]{/*}{*/},morecomment=[l]{\//},% 
   mathescape,
%   escapeinside={/*\%}{*/},%
   rangeprefix= /*< ,rangesuffix= >*/,%
   morestring=[d]{"}% 
 }
 
\lstset{breaklines=true,language=scala} 

\def\code{\lstinline}  % shorter version so you can write \code|String[Foo]|
                       % -- \def must be in same file as uses for this to
                       % work...
\newcommand{\lstref}[1]{Listing~\ref{#1}}
\newcommand{\Lstref}[1]{Listing~\ref{#1}} % only capitalise at beginning of sentence?
\newcommand{\secref}[1]{Section~\ref{#1}}
\newcommand{\Secref}[1]{Section~\ref{#1}} % only capitalise at beginning of sentence?



% \lstset{basicstyle=\footnotesize\ttfamily, breaklines=true, language=scala, tabsize=2, columns=fixed, mathescape=false,includerangemarker=false}
% thank you, Burak 
% (lstset tweaking stolen from
% http://lampsvn.epfl.ch/svn-repos/scala/scala/branches/typestate/docs/tstate-report/datasway.tex)
\lstset{
    xleftmargin=2em,%
    framesep=5pt,%
    frame=none,%
    captionpos=b,%
    fontadjust=true,%
    columns=[c]fixed,%
    keepspaces=false,%
    basewidth={0.56em, 0.52em},%
    tabsize=2,%
    basicstyle=\small\tt,% \small\tt
    commentstyle=\textit,%
    keywordstyle=\bfseries,%
    escapechar=\%,%
}

%% set latex/pdflatex specific stuff
%\ifpdf
    \usepackage[pdftex,
                hyperindex,
                plainpages=false,
                breaklinks,
                colorlinks,
                citecolor=black,
                filecolor=black,
                linkcolor=black,
                pagecolor=black,
                urlcolor=black]{hyperref}
    \usepackage[pdftex]{graphicx}
    \DeclareGraphicsExtensions{.jpg,.pdf}
    \pdfcatalog {
        /PageMode (/UseNone)
    }
    \usepackage{thumbpdf}
    \usepackage[pdftex]{color}
%\else
%    \usepackage[ps2pdf]{hyperref}
%    \usepackage{graphicx}
%    \DeclareGraphicsExtensions{.eps,.jpg}
%    \usepackage{color}
%\fi

%\setlength{\parindent}{0pt}
%\setlength{\parskip}{5pt}

% verbfilter stuff
\newcommand{\prog}[1]{{\sl #1}}
\newenvironment{program}[1][10.5]
  {\fontsize{#1}{13.6}\tt\begin{tabbing}\hspace*{0.5\parindent}\=\+\kill}
  {\end{tabbing}\noindent}
\newcommand{\blockcomment}[1]{{\color{grayPoint3}#1}}
\newcommand{\linecomment}{\color{grayPoint3}}
\newcommand{\grey}{\color{grey}}

%\newenvironment{program}{\ \ \ \ \begin{minipage}{\textwidth}\renewcommand{\baselinestretch}{1.0}\sl\begin{tabbing}}{\end{tabbing}\end{minipage}}
\newcommand{\vem}{\bfseries}
\newcommand{\quotedstring}[1]{{#1}}
\newcommand{\typename}[1]{{#1}}
\newcommand{\literal}[1]{{#1}}
\newcommand{\mi}[1]{\mathit{#1}}

% comments and notes
\newcommand{\comment}[1]{}
%\newcommand{\note}[1]{{\bf $\clubsuit$ #1 $\spadesuit$}}

% figures
\newcommand{\figurebox}[1]
        {\fbox{\begin{minipage}{\textwidth} #1 \medskip\end{minipage}}}
%        {\fbox{\begin{minipage}{\textwidth}\begin{center} #1 \end{center}\medskip\end{minipage}}}
\newcommand{\boxfig}[3]
        {\begin{figure*}\figurebox{#3\caption{\label{fig:#1}#2}}\end{figure*}}
\newcommand{\figref}[1]
        {Figure~\ref{fig:#1}}

% typing rules (not used here)
\newcommand{\ttag}[1]{\mbox{\textsc{\small(#1)}}}
\newcommand{\infer}[3]{\mbox{#1 }\ba{c} #2 \\ \hline #3 \ea}
\newcommand{\irule}[2]{{\renewcommand{\arraystretch}{1.2}\ba{c} #1 
                        \\ \hline #2 \ea}}
\newlength{\trulemargin}
\newlength{\trulewidth}
\newlength{\srulewidth}
\setlength{\trulemargin}{0.80cm}
\setlength{\trulewidth}{40.0mm}
\setlength{\srulewidth}{3.0cm}
\newenvironment{trules}{$\vspace{0.5em}\ba{p{\trulemargin}@{~}p{\trulewidth}@{~}p{\trulemargin}}}{\ea$}
\newenvironment{srules}{$\vspace{0.5em}\ba{p{\trulemargin}@{~}p{\srulewidth}}}{\ea$}
\newcommand{\laxiom}[2]{\ttag{#1} & $ #2 \hfill\ }
\newcommand{\raxiom}[2]{\hfill #2 $& \hfill \ttag{#1}}
\newcommand{\caxiom}[2]{\ttag{#1} & $\hfill #2 \hfill $& \ }
\newcommand{\lrule}[3]{\laxiom{#1}{\irule{#2}{#3}}}
\newcommand{\rrule}[3]{\raxiom{#1}{\irule{#2}{#3}}}
\newcommand{\crule}[3]{\caxiom{#1}{\irule{#2}{#3}}}
\newcommand{\lsrule}[3]{\lsaxiom{#1}{\irule{#2}{#3}}}
\newcommand{\rsrule}[3]{\rsaxiom{#1}{\irule{#2}{#3}}}
\newcommand{\nl}{\end{trules}\\[0.5em] \begin{trules}}
\newcommand{\snl}{\end{srules}\\[0.5em] \begin{srules}}

% commas and semicolons
\newcommand{\comma}{,\,}
\newcommand{\commadots}{\comma \ldots \comma}
\newcommand{\semi}{;\mbox{;};}
\newcommand{\semidots}{\semi \ldots \semi}

% spacing
\newcommand{\gap}{\quad\quad}
\newcommand{\biggap}{\quad\quad\quad}
\newcommand{\nextline}{\\ \\}
\newcommand{\htabwidth}{0.5cm}
\newcommand{\tabwidth}{1cm}
\newcommand{\htab}{\hspace{\htabwidth}}
\newcommand{\tab}{\hspace{\tabwidth}}
\newcommand{\linesep}{\ \hrulefill \ \smallskip}

% math stuff
\newenvironment{myproof}{{\em Proof:}}{$\Box$}
\newenvironment{proofsketch}{{\em Proof Sketch:}}{$\Box$}
\newcommand{\Case}{{\em Case\ }}

% make ; a delimiter in math mode
% \mathcode`\;="8000 % Makes ; active in math mode
% {\catcode`\;=\active \gdef;{\;}}
% \mathchardef\semicolon="003B

% reserved words
\newcommand{\mathem}{\bf}

% brackets
\newcommand{\set}[1]{\{#1\}}
\newcommand{\sbs}[1]{\lquote #1 \rquote}

% arrays
\newcommand{\ba}{\begin{array}}
\newcommand{\ea}{\end{array}}
\newcommand{\bda}{\[\ba}
\newcommand{\eda}{\ea\]}
\newcommand{\ei}{\end{array}}
\newcommand{\bcases}{\left\{\begin{array}{ll}}
\newcommand{\ecases}{\end{array}\right.}

% \cal ids
\renewcommand{\AA}{{\cal A}}
\newcommand{\BB}{{\cal B}}
\newcommand{\CC}{{\cal C}}
\newcommand{\DD}{{\cal D}}
\newcommand{\EE}{{\cal E}}
\newcommand{\FF}{{\cal F}}
\newcommand{\GG}{{\cal G}}
\newcommand{\HH}{{\cal H}}
\newcommand{\II}{{\cal I}}
\newcommand{\JJ}{{\cal J}}
\newcommand{\KK}{{\cal K}}
\newcommand{\LL}{{\cal L}}
\newcommand{\MM}{{\cal M}}
\newcommand{\NN}{{\cal N}}
\newcommand{\OO}{{\cal O}}
\newcommand{\PP}{{\cal P}}
\newcommand{\QQ}{{\cal Q}}
\newcommand{\RR}{{\cal R}}
\newcommand{\TT}{{\cal T}}
\newcommand{\UU}{{\cal U}}
\newcommand{\VV}{{\cal V}}
\newcommand{\WW}{{\cal W}}
\newcommand{\XX}{{\cal X}}
\newcommand{\YY}{{\cal Y}}
\newcommand{\ZZ}{{\cal Z}}

% misc symbols
\newcommand{\dhd}{\!\!\!\!\!\rightarrow}
\newcommand{\Dhd}{\!\!\!\!\!\Rightarrow}
\newcommand{\ts}{\,\vdash\,}
\newcommand{\la}{\langle}
\newcommand{\ra}{\rangle}
\newcommand{\eg}{{\em e.g.}}

% misc identifiers
\newcommand{\dom}{\mbox{\sl dom}}
\newcommand{\fn}{\mbox{\sl fn}}
\newcommand{\bn}{\mbox{\sl bn}}
\newcommand{\sig}{\mbox{\sl sig}}
\newcommand{\IF}{\mbox{\mathem if}}
\newcommand{\OTHERWISE}{\mbox{\mathem otherwise}}
\newcommand{\expand}{\prec}
\newcommand{\weakexpand}{\prec^W}
\newcommand{\spcomma}{~,~}
\newcommand{\sat}{{\bf\;sat\;}}

%\newcommand{\inst}{\mbox{\mathem inst}}
\newcommand{\trans}[1]{\la\!\la#1\ra\!\ra}
\newcommand{\remark}[1]{{\bf $\clubsuit$ #1 $\spadesuit$}}
\newcommand{\todo}[1]{\remark{to do: #1}}
%\newcommand{\J}{\justifies}
%\newcommand{\U}{\using}

% names
\newcommand{\Scala}{\mbox{\textsc{Scala}}}
\newcommand{\Java}{\mbox{\textsc{Java}}}

%\renewcommand\textfraction{.05}
%\renewcommand\floatpagefraction{.9}
%\renewcommand\topfraction{.8}

%%%%%%%%%%%%%%%%%%%%%%%%%%%%%%%%%%%%%%%
%   Language abstraction commands     %
%%%%%%%%%%%%%%%%%%%%%%%%%%%%%%%%%%%%%%%

%% Relations
% Subtype 
\newcommand{\sub}{<:}
% Type assignment
\newcommand{\typ}{:}
% reduction
\newcommand{\reduces}{\;\rightarrow\;}
% well-formedness
\newcommand{\wf}{\;\mbox{\textbf{wf}}}
\newcommand{\wfe}{\;\mbox{\textbf{wfe}}}

%% Operators
% Type selection
\newcommand{\tsel}{\#}
% Function type
\newcommand{\tfun}{\rightarrow}
\newcommand{\dfun}[3]{(#1\!:\!#2) \Rightarrow #3}
% Conjunction
\newcommand{\tand}{\wedge}
% Disjunction
\newcommand{\tor}{\vee}
% Singleton type suffix
\newcommand{\sing}{.\textbf{type}}

%% Syntax
% Header for typing rules
\newcommand{\judgement}[2]{{\bf #1} \hfill #2}
% Refinement
\newcommand{\refine}[2]{\left\{#1 \Rightarrow #2 \right\}}
\newcommand{\mlrefine}[2]{\{#1 \Rightarrow #2 \}}
% Field definitions
\newcommand{\ldefs}[1]{\left\{#1\right\}}
\newcommand{\mlldefs}[1]{\{#1\}}
% Member sequences
\newcommand{\seq}[1]{\overline{#1}}
% Lambda
\newcommand{\dabs}[3]{(#1\!:\!#2)\Rightarrow #3}
\newcommand{\abs}[3]{\lambda #1\!:\!#2.#3}
% Method Application
\newcommand{\mapp}[3]{#1.#2(#3)}
% Substitution
\newcommand{\subst}[3]{[#1/#2]#3}
% Object creation
\newcommand{\new}[3]{\textbf{val }#1 = \textbf{new }#2 ;\; #3}
\newcommand{\mlnew}[3]{\textbf{val }#1 = \textbf{new }#2 ;\;\\&#3}
%\renewcommand{\new}[3]{#1 \leftarrow #2 \,\textbf{in}\, #3}
% Field declaration
\newcommand{\Ldecl}[3]{#1 : #2..#3}%{#1 \operatorname{>:} #2 \operatorname{<:} #3}
\newcommand{\ldecl}[2]{#1 : #2}
\newcommand{\mdecl}[3]{#1 : #2 \tfun #3}
% Top and Bottom
\newcommand{\Top}{\top}%{\textbf{Top}}
\newcommand{\Bot}{\bot}%\textbf{Bot}}
% Environment extension
%\newcommand{\envplus}[1]{\uplus \{ #1 \}}
\newcommand{\envplus}[1]{, #1}
% Reduction
\newcommand{\reduction}[4]{#1 \operatorname{|} #2 \reduces #3 \operatorname{|} #4}

% Sugar
\newcommand{\arrow}[2]{#1\rightarrow_s#2}
\newcommand{\fun}[4]{\textbf{fun } (#1:#2)\;#3\;#4}
\newcommand{\app}[2]{(\textbf{app }#1\;#2)}
\newcommand{\mlapp}[2]{(\textbf{app }#1\;\\&#2)}
\newcommand{\cast}[2]{(\textbf{cast }#1\;#2)}

\newcommand{\lindent}{\hspace{-4mm}}

% Logical relations
\newcommand{\relv}[4]{\mathcal{V}_{#1;#2;#3}\llbracket#4\rrbracket}
\newcommand{\rele}[4]{\mathcal{E}_{#1;#2;#3}\llbracket#4\rrbracket}
\newcommand{\rels}[3]{\mathcal{\supseteq}_{#1}\llbracket#2;#3\rrbracket}
\newcommand{\relg}[3]{\mathcal{\supseteq}_{#1;#2}\llbracket!#3\rrbracket}
\newcommand{\irred}[2]{\text{irred }(#1,#2)}
\newcommand{\andl}{\;\wedge\;}
\newcommand{\orl}{\vee}
\newcommand{\impliesl}{\rightarrow}
\newcommand{\reductionl}[5]{#1 \operatorname{|} #2 \;\rightarrow^{#5}\; #3 \operatorname{|} #4}
\newcommand{\ds}{\,\vDash\,}

\begin{document}

\conferenceinfo{FOOL '12}{October 22, 2012, Tucson, AZ, USA.} 
\copyrightyear{2012} 
\copyrightdata{[to be supplied]} 

\title{Dependent Object Types}
\subtitle{Towards a foundation for Scala's type system}

\authorinfo{Nada Amin \and Adriaan Moors \and Martin Odersky}
           {EPFL}
           {first.last@epfl.ch}

\maketitle

\begin{abstract}
We propose a new type-theoretic foundation of Scala and languages like
it: the Dependent Object Types calculus (DOT). DOT models Scala's
path-dependent types and abstract type members, as well as its mixture
of nominal and structural typing through the use of refinement
types. It makes no attempt to model inheritance or mixing
composition. The calculus does not model what's currently in Scala: it
is more normative than descriptive.

We show that DOT and its patched-up variants are not syntactically
sound, by exhibiting counterexamples to preservation. Nevertheless, we
sketch a proof of type-safety of the calculus via step-indexed logical
relations.
\end{abstract}

\category{D.3.3}{Language Constructs and Features}{Abstract data types, Classes and objects, polymorphism}
\category{D.3.1}{Formal Definitions and Theory}{Syntax, Semantics}
\category{F.3.1}{Specifying and Verifying and Reasoning about Programs}{}
\category{F.3.3}{Studies of Program Constructs}{Object-oriented constructs, type structure}
\category{F.3.2}{Semantics or Programming Languages}{Operational semantics}

\terms
Languages, Theory, Verification

\keywords
calculus, objects, dependent types, step-indexed logical relations

\section{Introduction}

This paper presents a proposal for a new type-theoretic foundation of
Scala and languages like it. The properties we are interested in
modeling are Scala's path-dependent types and abstract type members,
as well as its mixture of nominal and structural typing through the
use of refinement types. Compared to previous approaches~\cite{nuObj,FS},
we make no attempt to model inheritance or mixin composition. Indeed
we will argue that such concepts are better modeled in a different
setting.

The calculus does not precisely describe what's currently in Scala. It
is more normative than descriptive. The main point of deviation
concerns the difference between Scala's compound type formation using
{\bf with} and classical type intersection, as it is modeled in the
calculus. Scala, and the previous calculi attempting to model it
conflates the concepts of compound types (which inherit the members of
several parent types) and mixin composition (which build classes from
other classes and traits). At first glance, this offers an economy of
concepts. However, it is problematic because mixin composition and
intersection types have quite different properties. In the case of
several inherited members with the same name, mixin composition has to
pick one which overrides the others. It uses for that the concept of
linearization of a trait hierarchy. Typically, given two independent
traits $T_1$ and $T_2$ with a common method $m$, the mixin composition
\code@$T_1$ with $T_2$@ would pick the $m$ in $T_2$, whereas the member in
$T_1$ would be available via a super-call. All this makes sense from
an implementation standpoint. From a typing standpoint it is more
awkward, because it breaks commutativity and with it several
monotonicity properties.

In the present calculus, we replace Scala's compound types by
classical intersection types, which are commutative. We also
complement this by classical union types. Intersections and unions
form a lattice wrt subtyping. This addresses another problematic
feature of Scala: In Scala's current type system, least upper bounds
and greatest lower bounds do not always exist. Here is an example:
Given two traits
\begin{lstlisting}
  trait A { type T <: A }
  trait B { type T <: B }
\end{lstlisting}
The greatest lower bound of \code@A@ and \code@B@ is approximated by the
infinite sequence
\begin{lstlisting}
  A with B { type T <: A with B { type T <: A with B {
    type T < ...
  }}}
\end{lstlisting}
The limit of this sequence does not exist as a type in Scala.

This is problematic because glbs and lubs play a central role in
Scala's type inference. The absence of universal glbs and lubs makes
type inference more brittle and more unpredictable.

\subsection{Why No Inheritance?}

In the calculus we made the deliberate choice not to model any form of
inheritance. This is, first and foremost, to keep the calculus simple.
Secondly, there are many different approaches to inheritance and
mixin composition, so that it looks advantageous not to tie the basic
calculus to a specific one. Finally, it seems that the modelization of
inheritance lends itself to a different approach than the basic
calculus. For the latter, we need to prove type safety of the calculus.
One might try to do this also for a calculus with inheritance, but our
experience suggests that this complicates the proofs considerably.  An
alternative approach that might work better is to model inheritance
it as a form of code-reuse. Starting with an enriched type system with
inheritance, and a translation to the basic calculus, one needs to
show type safety wrt the translation. This might be easier than
to prove type safety wrt reduction.

\section{The DOT Calculus}

\boxfig{dot-one}{The DOT Calculus : Syntax, Reduction, Type Assignment}{
{\bf Syntax}\medskip
\begin{center}    
$\ba{l@{\hspace{0.2mm}}|@{\hspace{0.2mm}}l}
\ba[t]{l@{\hspace{10mm}}l}
x, y, z    & \lindent{\mbox{Variable}} \\
l          & \lindent{\mbox{Value label}}\\
m          & \lindent{\mbox{Method label}}\\[0.2em]
v ::=      & \lindent{\mbox{Value}} \\
\gap x     & \mbox{variable} \\[0.2em]
t ::=      & \lindent{\mbox{Term}} \\
\gap v     & \mbox{value} \\
\gap \new x c t & \mbox{new instance} \\
\gap t.l  & \mbox{field selection} \\
\gap \mapp t m t  & \mbox{method invocation} \\[0.2em]
p ::= & \lindent \mbox{Path} \\
\gap x & \mbox{variable} \\
\gap p.l & \mbox{selection} \\
c ::= T_c \ldefs{\seq{d}} & \lindent{\mbox{Constructor}} \\[0.2em]
d ::= & \lindent{\mbox{Initialization}}\\
\gap l = v & \mbox{field initialization}\\
\gap m(x) = t & \mbox{method initialization}\\
s      ::= \seq{x \mapsto c} & \lindent\mbox{Store}
\ea
&
\ba[t]{l@{\hspace{10mm}}l}
L ::=      & \lindent{\mbox{Type label}} \\
\gap L_c   & \mbox{class label} \\
\gap L_a   & \mbox{abstract type label} \\[0.2em]
S,T,U,V,W ::= & \lindent\mbox{Type}\\
\gap p.L & \mbox{type selection} \\
\gap T \refine z {\seq D} & \mbox{refinement} \\
\gap T \tand T & \mbox{intersection type} \\
\gap T \tor T & \mbox{union type} \\
\gap \Top  & \mbox{top type} \\
\gap \Bot  & \mbox{bottom type} \\[0.2em]
S_c, T_c ::= & \lindent \mbox{Concrete type} \\
\multicolumn{2}{l}{\gap p.L_c ~|~ T_c \refine z {\seq D} ~|~ T_c \wedge T_c  ~|~ \Top} \\[0.2em]
D ::= & \lindent\mbox{Declaration} \\
\gap \Ldecl L S U & \mbox{type declaration} \\
\gap \ldecl l T   & \mbox{value declaration} \\
\gap \mdecl m S U & \mbox{method declaration}\\
\ \\
\Gamma ::= \seq{x \typ T} & \lindent\mbox{Environment}
\ea
\ea$
\end{center}
\medskip

\linesep

\begin{multicols}{2}[\judgement{Reduction}{\fbox{$\reduction t s {t'} {s'}$}}]

\infrule[\textsc{msel}]
{y \mapsto T_c \ldefs{\seq{l = v'}\;\seq{m(x)=t}} \in s}
{\reduction {\mapp y {m_i} v} s {\subst v {x_i} {t_i}} s}

\infrule[\textsc{sel}]
{y \mapsto T_c \ldefs{\seq{l = v}\;\seq{m(x)=t}} \in s}
{\reduction {y.l_i} s {v_i} s}

\infax[\textsc{new}]
{\reduction {\new x c t} s t {s \envplus{x \mapsto c}}}

\infrule[\textsc{context}]
{\reduction t s {t'} {s'}}
{\reduction {e[t]} s {e[t']} s'}
\end{multicols}

\hfill {\bf where} evaluation context $\gap e ::= [\,] ~|~ \mapp e m t ~|~ \mapp v m e ~|~ e.l\hspace{2cm}$

\linesep

\begin{multicols}{2}[\judgement{Type Assignment}{\fbox{$\Gamma \ts t \typ T$}}]

\infrule[\textsc{var}]
{x \typ T \in \Gamma}
{\Gamma \ts x \typ T}

\infrule[\textsc{msel}]
{\Gamma \ts t \ni m \typ {S \tfun T} \\
 \Gamma \ts t' \typ T' \spcomma T' \sub S}
{\Gamma \ts \mapp t m {t'} \typ T}

\infrule[\textsc{sel}]
{\Gamma \ts t \ni l \typ T'}
{\Gamma \ts t.l \typ T'}

\infrule[\textsc{new}]
{
y \notin \fn(T') \\
\Gamma \ts T_c \wfe \spcomma T_c \expand_y \seq{D} \\
\Gamma \envplus{y: T_c} \ts \seq{D \sat \seq{d}} \spcomma {t'} \typ {T'}}
{\Gamma \ts \new y {T_c \ldefs{\seq{d}}} {t'} \typ T'}

\end{multicols}

\linesep

\begin{multicols}{2}[\judgement{Satisfied Declaration}{\fbox{$\Gamma \ts D \sat \seq{d}$}}]

\infrule[\textsc{tdecl-sat}]
{\Gamma \ts S \sub U}
{\Gamma \ts \Ldecl L S U \sat \seq{d}}

\infrule[\textsc{vdecl-sat}]
{\Gamma \ts v \typ {V'} \spcomma {V'} \sub V}
{\Gamma \ts \ldecl l V \sat l = v,\seq{d}}

\infrule[\textsc{mdecl-sat}]
{\Gamma \ts S \wfe\\
 \Gamma \envplus{x : S} \ts t \typ {T'} \spcomma {T'} \sub T}
{\Gamma \ts \mdecl m S T \sat m(x) = t,\seq{d}}

\end{multicols}

} % END dot-one

The DOT calculus is a simple system of dependent
object-types. \figref{dot-one} gives its syntax, reduction rules,
and type assignment rules.

\subsection{Notation} We use standard notational conventions for
sets. The notation $\seq{X}$ denotes a set of elements $X$. Given a
such a set $\seq X$ in a typing rule, $X_i$ denotes an arbitrary
element of $X$. 
%The $\uplus$ operator extends a set of bindings. It is required that the added binding does not
%introduce a variable which is already bound in the base-set.
We use an
abbreviation for preconditions in typing judgements. Given an
environment $\Gamma$ and some predicates $P$ and $Q$, the condition $\Gamma \ts P \spcomma Q$
is a shorthand for the two conditions $\Gamma \ts P$ and $\Gamma \ts Q$.

\subsection{Syntax}

There are four alphabets: Variable names $x$, $y$, $z$ are freely
alpha-renamable. They occur as parameters of methods, as
binders for objects created by \verb@new@-expressions, and as self
references in refinements. Value labels $l$ denote fields in objects,
which are bound to values at run-time. Similarly, method labels $m$
denote methods in objects. Type labels $L$ denote type members of
objects. Type labels are further separated into labels for abstract
types $L_a$ and labels for classes $L_c$. It is assumed that in each
program every class label $L_c$ is declared at most once.

We assume that the label alphabets $l$, $m$ and $L$ are finite. This is
not a restriction in practice, because one can include in these 
alphabets every label occurring in a given program.

The terms $t$ in DOT consist of variables $x$, $y$, $z$, field
selections $t.l$, method invocations $t.m(t)$ and object creation
expressions $\new y c {t'}$ where $c$ is a constructor $T_c \ldefs{\seq{l
    = v}\;\seq{m(x) = t}}$. The latter binds a variable $y$ to a new
instance of type $T_c$ with fields $\seq l$ initialized to values
$\seq v$ and methods $\seq m$ initialized to methods of one parameter
$\seq{x}$ and body $\seq{t}$.  The scope of $y$ extends through the term
${t'}$.

Two subclasses of terms are values $v$, which consist of just
variables, and paths $p$ which consist of just variables and field
selections.

The types in DOT are denoted by letters $S$, $T$, $U$, $V$, or $W$. They consist of the following:
\begin{itemize}
\item[-] Type selections $p.L$, which denote the type member $L$ of path $p$.
\item[-] Refinement types $T \refine z {\seq D}$, which refine a type $T$ by a set of declarations $D$.
         The variable $z$ refers to the ``self''-reference of the type. Declarations can refer to
         other declarations in the same type by selecting from $z$.
\item[-] Type intersections $T \tand T'$, which carry the declarations of members present in either $T$ or $T'$.
\item[-] Type unions $T \tor T'$, which carry only the declarations of members present in both $T$ and $T'$.
\item[-] A top type $\Top$, which corresponds to an empty object.
\item[-] A bottom type $\Bot$, which represents a non-terminating computation.
\end{itemize}
A subset of types $T_c$ are called {\em concrete types}. These are type selections
$p.L_c$ of class labels,
the top type $\Top$, intersections of concrete types, and refinements $T_c \refine z {\seq D}$ of concrete types. Only concrete types are allowed in constructors $c$.

There are only three forms of declarations in DOT, which are all part
of refinement types.  A value declaration $\ldecl l T$ introduces a
field with type $T$.  A method declaration $\mdecl m S U$ introduces a
method with parameter of type $S$ and result of type $U$. A type
declaration $\Ldecl L S U$ introduces a type member $L$ with a lower
bound type $S$ and an upper bound type $U$. There are no type aliases,
but a type alias can be simulated by a type declaration $\Ldecl L T T$
where the lower bound and the upper bound are the same type $T$.

\boxfig{dot-decls}{The DOT Calculus : Declaration Lattice}{

\bda{lcl@{\gap}l}
    
      \dom(\seq D \tand \seq {D'}) &~=~& \dom(\seq{D}) \cup \dom(\seq{D'}) \\
      \dom(\seq D \tor \seq {D'}) &=& \dom(\seq{D}) \cap \dom(\seq{D'}) \\[0.5em]
      (D \tand D')(L) &=&
        \Ldecl L {(S \tor S')} {(U \tand U')} & \mbox{if~} (\Ldecl L S U) \in \seq{D} \;\mbox{and}\; (\Ldecl L {S'} {U'}) \in \seq{D'} \\
        &=& D(L) & \mbox{if~} L \notin \dom(\seq{D'}) \\
        &=& D'(L) & \mbox{if~} L \notin \dom(\seq{D}) \\
      (D \tand D')(m) &=&
        \mdecl m {(S \tor S')} {(U \tand U')} & \mbox{if~} (\mdecl m S U) \in \seq{D} \;\mbox{and}\; (\mdecl m {S'} {U'}) \in \seq{D'} \\
        &=& D(m) & \mbox{if~} m \notin \dom(\seq{D'}) \\
        &=& D'(m) & \mbox{if~} m \notin \dom(\seq{D}) \\
      (D \tand D')(l) &=&
        \ldecl l {T \tand T'} & \mbox{if~} (\ldecl l T) \in \seq{D} \;\mbox{and}\; (\ldecl l {T'}) \in \seq{D'} \\
        &=& D(l) & \mbox{if~} l \notin \dom(\seq{D'}) \\
        &=& D'(l) & \mbox{if~} l \notin \dom(\seq{D}) \\[0.5em]
      (D \tor D')(L) &=&
        \Ldecl L {(S \tand S')} {(U \tor U')} & \mbox{if~} (\Ldecl L S U) \in \seq{D} \;\mbox{and}\; (\Ldecl L {S'} {U'}) \in \seq{D'} \\
      (D \tor D')(m) &=&
        \mdecl m {(S \tand S')} {(U \tor U')} & \mbox{if~} (\mdecl m S U) \in \seq{D} \;\mbox{and}\; (\mdecl m {S'} {U'}) \in \seq{D'} \\
      (D \tor D')(l) &=&
        \ldecl l {T \tor T'} & \mbox{if~} (\ldecl l T) \in \seq{D} \;\mbox{and}\; (\ldecl l {T'}) \in \seq{D'} 
    \eda

Sets of declarations form a lattice with the given meet $\wedge$ and
join $\vee$, the empty set of declarations as the top element, and the
bottom element $\seq{D_\Bot}$, Here $\seq{D_\Bot}$ is the set of
declarations that contains for every term label $l$ the declaration
$\ldecl l \Bot$, for every type label $L$ the declaration $\Ldecl L
\Top \Bot$ and for every method label $m$ the declaration $\mdecl m
\Top \Bot$.
}

Every field or type label can be declared only once in a set of
declarations $\seq D$. A set of declarations can hence be seen as a map from
labels to their declarations.  Meets $\wedge$ and joins $\vee$ on sets of
declarations are defined in ~\figref{dot-decls}.

\subsection{Reduction rules}

Reduction rules $\reduction t s {t'} {s'}$ in DOT rewrite pairs of
terms $t$ and stores $s$, where stores map variables to constructors.
There are three main reduction rules: Rule (\textsc{msel}) rewrites a
method invocation $\mapp y {m_i} v$ by retrieving the corresponding method
definition from the store, and performing a substitution of the
argument for the parameter in the body. Rule (\textsc{sel}) rewrites a
field selection $x.l$ by retrieving the corresponding value from the
store. Rule (\textsc{new}) rewrites an object creation $\new x c t$ by
placing the binding of variable $x$ to constructor $c$ in the store
and continuing with term $t$.  These reduction rules can be applied
anywhere in a term where the hole $[\,]$ of an evaluation context $e$
can be situated.

\subsection{Type assignment rules}

The last part of \figref{dot-one} presents rules for type
assignment.  

Rules (\textsc{sel}) and (\textsc{msel}) type field selections and
method invocations by means of an auxiliary membership relation $\ni$,
which determines whether a given term contains a given declaration as
one of its members. The membership relation is defined in
\figref{dot-two} and is further explained in section~\ref{mem}.

The last rule, (\textsc{new}), assigns types to object creation
expressions. It is the most complex of DOT's typing rules.  To
type-check an object creation $\new y {T_c \ldefs {\seq{l =
      v}\;\seq{m(x) = t}}} t'$, one verifies first that the type $T_c$
is well-formed (see \figref{dot-wf} for a definition of
well-formedness).  One then determines the set of all declarations
that this type carries, using the expansion relation $\expand$ defined
in \figref{dot-two}.  Every type declaration $\Ldecl L S U$ in
this set must be realizable, i.e.\ its lower bound $S$ must be a
subtype of its upper bound $U$.  Every field declaration $\ldecl l V$
in this set must have a corresponding initializing value of $v$ of
type $V$.  These checks are made in an environment which is extended
by the binding $y: T_c$. In particular this allows field values that
recurse on ``self'' by referring to the bound variable $x$. Similarly,
every method declaration $\mdecl m T W$ must have a corresponding
initializing method definition $m(x) = t$. The parameter type $T$ must
be $\wfe$ (well-formed and expanding; see \figref{dot-wf}), and
the body $t$ must type check to $W$ in an environment extended by the
bindings $y : T_c$ and $x : T$.

Instead of adding a separate subsumption rule, subtyping is expressed
by preconditions in rules (\textsc{msel}) and (\textsc{new}).

\boxfig{dot-two}{The DOT Calculus : Membership and Expansion}{
    \begin{multicols}{2}[\judgement{Membership}{\fbox{$\Gamma \ts t \ni D$}}]

      \infrule[\textsc{path-$\ni$}]
      {\Gamma \ts p \typ T \spcomma T \expand_z \seq D}
      {\Gamma \ts p \ni \subst p z {D_i}}

      \infrule[\textsc{term-$\ni$}]
      {z \not\in \fn(D_i) \andalso \Gamma \ts t \typ T \spcomma T \expand_z \seq D}
      {\Gamma \ts t \ni D_i}
   \end{multicols}

   \linesep

    \begin{multicols}{2}[\judgement{Expansion}{\fbox{$\Gamma \ts T \expand_z \seq D$}}]  

      \infrule[\textsc{rfn-$\expand$}]
      {\Gamma \ts T \expand_z {\seq D'}}
      {\Gamma \ts T \refine z {\seq D} \expand_z \seq {D'} \tand \seq D}

      \infrule[\textsc{$\tand$-$\expand$}]
      {\Gamma \ts T_1 \expand_z {\seq D_1} \spcomma T_2 \expand_z {\seq D_2}}
      {\Gamma \ts T_1 \tand T_2 \expand_z {\seq D_1 \tand \seq D_2}}

      \infax[\textsc{$\Top$-$\expand$}]
      {\Gamma \ts \Top \expand_z \{\}}

      \infrule[\textsc{tsel-$\expand$}]
      {\Gamma \ts p \ni \Ldecl L S U \spcomma U \expand_z \seq D}
      {\Gamma \ts p.L \expand_z \seq D}

      \infrule[\textsc{$\tor$-$\expand$}]
      {\Gamma \ts T_1 \expand_z {\seq D_1} \spcomma T_2 \expand_z {\seq D_2}}
      {\Gamma \ts T_1 \tor T_2 \expand_z {\seq D_1 \tor \seq D_2}}

      \infax[\textsc{$\Bot$-$\expand$}]
      {\Gamma \ts \Bot \expand_z \seq{D_\Bot}}
    \end{multicols}

} % END dot-two

\subsection{Membership}\label{mem}

\figref{dot-two} presents typing rules for membership and expansion.
The membership judgement $\Gamma \ts t \ni D$ states that in
environment $\Gamma$ a term $t$ has a declaration $D$ as a member. The
membership rules rely on expansion. There are two rules, one for paths
(\textsc{path-$\ni$}) and one for general terms
(\textsc{term-$\ni$}). For general terms, the ``self''-reference of
the type must not occur in the resulting declaration $D$, since, to
guarantee syntactic validity, we can only substitute a path for the
``self''-reference.

\subsection{Expansion}
      
The expansion relation $\expand$ is needed to typecheck the complete
set of declarations carried by a concrete type that is used in a
\textbf{new}-expression. Expansion is also used by the membership
rules and in subtyping refinements on the right (see
\figref{dot-sub}).

Rule (\textsc{rfn-$\expand$}) states that a refinement type $T
\expand_z {\seq D}$ expands to the conjunction of the expansion $D'$
of $T$ and the newly added declarations $D$. Rule
(\textsc{tsel-$\expand$}) states that a type selection $p.L$ carries
the same declarations as the upper bound $U$ of $L$ in $T$.  Rules
($\tand$-$\expand$) and ($\tor$-$\expand$) states that expansion
distributes through meets and joins.  Rule (\textsc{$\top$-$\expand$})
states that the top type $\top$ expands to the empty set. Rule
(\textsc{$\bot$-$\expand$}) states that the bottom type $\bot$ expands
to the bottom element $\seq{D_\Bot}$ of the lattice of sets of
declarations (recall \figref{dot-decls}).

\boxfig{dot-sub}{The DOT Calculus : Subtyping and Declaration Subsumption}{

    \begin{multicols}{2}[\judgement{Subtyping}{\fbox{$\Gamma \ts S \sub T$}}]

%      \infrule[\textsc{trans}]
%      {\Gamma \ts S \sub T \spcomma T \sub U}
%      {\Gamma \ts S \sub U}

      \infrule[\textsc{refl}]
      {\Gamma \ts T \wfe}
      {\Gamma \ts T \sub T}

      \infrule[\textsc{$\sub$-rfn}]
      {\Gamma \ts {T \refine z {\seq D}} \wfe
       \spcomma S \sub T \spcomma S \expand_z \seq{D'} \\
       \Gamma \envplus{z: S} \ts \seq{D' \sub D}}
      {\Gamma \ts S \sub T \refine z {\seq D}}

      \infrule[\textsc{$\sub$-tsel}]
      {\Gamma \ts p \ni \Ldecl L S U \spcomma S \sub U \spcomma S' \sub S }
      {\Gamma \ts S' \sub p.L}

      \infrule[\textsc{$\sub$-$\tand$}]
      {\Gamma \ts T \sub T_1 \spcomma T \sub T_2}
      {\Gamma \ts T \sub T_1 \tand T_2}

      \infrule[\textsc{$\sub$-$\tor_1$}]
      {\Gamma \ts T_2 \wfe \spcomma T \sub T_1}
      {\Gamma \ts T \sub T_1 \tor T_2}

      \infrule[\textsc{$\sub$-$\tor_2$}]
      {\Gamma \ts T_1 \wfe \spcomma T \sub T_2}
      {\Gamma \ts T \sub T_1 \tor T_2}
      
      %%%%%%%%%%%%%%%%%%%%%%%%%%%%%%%%%

      \infrule[\textsc{$\sub$-$\Top$}]
      {\Gamma \ts T \wfe}
      {\Gamma \ts T \sub \Top}

      \infrule[\textsc{$\Bot$-$\sub$}]
      {\Gamma \ts T \wfe}
      {\Gamma \ts \Bot \sub T}

      \infrule[\textsc{rfn-$\sub$}]
      {\Gamma \ts {T \refine z {\seq D}} \wfe
       \spcomma T \sub T'}
      {\Gamma \ts T \refine z {\seq D} \sub T'}

      \infrule[\textsc{tsel-$\sub$}]
      {\Gamma \ts p \ni \Ldecl L S U \spcomma S \sub U \spcomma U \sub U'}
      {\Gamma \ts p.L \sub U'}

      \infrule[\textsc{$\tor$-$\sub$}]
      {\Gamma \ts T_1 \sub T \spcomma T_2 \sub T}
      {\Gamma \ts T_1 \tor T_2 \sub T}

      \infrule[\textsc{$\tand_1$-$\sub$}]
      {\Gamma \ts T_2 \wfe \spcomma T_1 \sub T}
      {\Gamma \ts T_1 \tand T_2 \sub T}

      \infrule[\textsc{$\tand_2$-$\sub$}]
      {\Gamma \ts T_1 \wfe \spcomma T_2 \sub T}
      {\Gamma \ts T_1 \tand T_2 \sub T}

      \end{multicols}

    \linesep

    \begin{multicols}{2}[\judgement{Declaration subsumption}{\fbox{$\Gamma \ts D \sub D'$}}]

    \infrule[\textsc{tdecl-$\sub$}]
            {\Gamma \ts S' \sub S \spcomma T \sub T'}
            {\Gamma \ts (\Ldecl L S T) \sub (\Ldecl L {S'} {T'})}

    \infrule[\textsc{vdecl-$\sub$}]
            {\Gamma \ts T \sub T'}
            {\Gamma \ts (\ldecl l T) \sub (\ldecl l {T'})}

    \infrule[\textsc{mdecl-$\sub$}]
            {\Gamma \ts S' \sub S \spcomma T \sub T'}
            {\Gamma \ts (\mdecl m S T) \sub (\mdecl m {S'} {T'})}

    \end{multicols}
} % END dot-sub

\subsection{Subtyping}

\figref{dot-sub} defines the subtyping judgement $\Gamma \ts S
\sub T$ which states that in environment $\Gamma$ type $S$ is a
subtype of type $T$. Subtyping is regular wrt $\wfe$: if
type $S$ is a subtype of type $T$, then $S$ and $T$ are well-formed
and expanding. Though this regularity limits our calculus to
$\wfe$-types, this limitation allows us to show that subtyping is
transitive, as discussed in section~\ref{dot-preservation}.

\subsection{Declaration Subsumption}

The declaration subsumption judgement $\Gamma \ts D \sub D'$ in
\figref{dot-sub} states that in environment $\Gamma$ the declaration
$D$ subsumes the declaration $D'$. There are three rules, one for each
kind (type, value, method) of declarations.  Rule
(\textsc{tdecl-$\sub$}) states that a type declaration $\Ldecl L S U$
subsumes another type declaration $\Ldecl L {S'} {U'}$ if $S'$ is a
subtype of $S$ and $U$ is a subtype of $U'$. In other words, the set
of types between $S$ and $U$ is contained in the set of types between
$S'$ and $U'$.  Rule (\textsc{vdecl-$\sub$}) states that a value
declaration $\ldecl l T$ subsumes another value declaration $\ldecl l
T'$ if $T$ is a subtype of $T'$. Rule (\textsc{mdecl-$\sub$}) is
similar to (\textsc{tdecl-$\sub$}), as the parameter type varies
contravariantly and the return type covariantly.

Declaration subsumption is extended to a binary relation between
sequences of declarations as follows: $\seq{D} <: \seq{D'}$ iff
$\forall D'_i \exists D_j. D_j \sub D'_i$.

\boxfig{dot-wf}{The DOT Calculus : Well-Formedness}{

       \begin{multicols}{2}[\judgement{Well-formed types}{\fbox{$\Gamma \ts T \wf$}}]

      \infrule[\textsc{rfn-wf}]
      {\Gamma \ts T \wf \\ 
       \Gamma \envplus {z: T \refine z {\seq D}} \ts \seq {D \wf}}
      {\Gamma \ts T \refine z {\seq D} \wf}

      \infrule[\textsc{tsel-wf$_1$}]
      {\Gamma \ts p \ni \Ldecl L S U \spcomma S \wf \spcomma U \wf}
      {\Gamma \ts p.L \wf}

      \infrule[\textsc{$\tand$-wf}]
      {\Gamma \ts T \wf \spcomma T' \wf}
      {\Gamma \ts T \tand T' \wf}

      %%%%%%%%%%%%%%%%%%%%%%%%%%%

      \infax[\textsc{$\Top$-wf}]
      {\Gamma \ts \Top \wf}

      \infax[\textsc{$\Bot$-wf}]
      {\Gamma \ts \Bot \wf}

      \infrule[\textsc{tsel-wf$_2$}]
      {\Gamma \ts p \ni \Ldecl L \Bot U}
      {\Gamma \ts p.L \wf}

      \infrule[\textsc{$\tor$-wf}]
      {\Gamma \ts T \wf \spcomma T' \wf}
      {\Gamma \ts T \tor T' \wf}

    \end{multicols}

    \linesep

    \begin{multicols}{2}[\judgement{Well-formed declarations}{\fbox{$\Gamma \ts D \wf$}}]
      \infrule[\textsc{tdecl-wf}]
      {\Gamma \ts S \wf \spcomma U \wf}
      {\Gamma \ts \Ldecl L S U \wf}

      \infrule[\textsc{vdecl-wf}]
      {\Gamma \ts T \wf}
      {\Gamma \ts \ldecl l T \wf}

      \infrule[\textsc{mdecl-wf}]
      {\Gamma \ts S \wf \spcomma U \wf}
      {\Gamma \ts \mdecl m S U \wf}

    \end{multicols}

    \linesep

    \begin{multicols}{2}[\judgement{Well-formed and expanding types}{\fbox{$\Gamma \ts T \wfe$}}]
      \infrule[\textsc{wfe}]
      {\Gamma \ts T \wf \spcomma T \expand_z \seq{D}}
      {\Gamma \ts T \wfe}
    \end{multicols}
} % END dot-wf

\subsection{Well-formedness}

The well-formedness judgement $\Gamma \ts T \wf$ in \figref{dot-wf}
states that in environment $\Gamma$ the type $T$ is
well-formed.

A refinement type $T \refine z {\seq D}$ is
well-formed if the parent type $T$ is well-formed and every
declaration in $\seq D$ is well-formed in an environment augmented by
the binding of the self-reference $z$ to the refinement type itself
(\textsc{rfn-wf}).

A type selection $p.L$ is well-formed if $L$ is a member of $p$, and
the lower bound of $L$ is also well-formed (\textsc{tsel-wf$_1$} and
\textsc{tsel-wf$_2$}). The latter condition has the effect that the
lower bound of a type $p.L$ may not refer directly or indirectly to a
type containing $p.L$ itself --- if it would, the well-formedness
judgement of $p.L$ would not have a finite proof. No such restriction
exists for the upper bound of $L$ if the lower bound is $\bot$
(\textsc{tsel-wf$_2$}). The upper bound may in fact refer back to the
type. Hence, recursive class types and F-bounded abstract types are
both expressible.

The other forms of types in DOT are all well-formed if their
constituent types are well-formed.

Well-formedness extends straightforwardly to declarations with the
judgement $\Gamma \ts D \wf$. All declarations are well-formed if
their constituent types are well-formed.

\section{Examples}\label{dot-examples}

\subsection{Sugar: Functions}
Like in Scala, we can encode functions as objects with a special
method. We will freely use the following sugar in the remaining of
this paper. Note that the variable $z$ must be fresh.
\begin{align*}
\arrow S T &\iff {\Top \refine z {\mdecl {\mi{apply}} S T}}\\
\fun x S T t &\iff \new z {{\arrow S T}\ldefs{\mi{apply}(x) = t}} z\\
\app f x &\iff \mapp f {\mi{apply}} x\\
\cast T t&\iff \app {\fun x T T x} t
\end{align*}

In the remaining examples, for brevity, we will also use $\abs x S t$
for $\fun x S \_ t$ where the return type can be somewhat inferred
from the context.

\subsection{Basics: Booleans, Error, \ldots}
This program defines a $\mi{root}$ object with basic types
($\mi{Unit}$, $\mi{Boolean}$, $\mi{Nat}$) and methods ($\mi{true}$,
$\mi{false}$, $\mi{error}$, $\mi{zero}$). For brevity, we've omitted
the code related to natural numbers. During the object creation, the
method labels such as $\mi{false}$ are all initialized.
\begin{align*}
&\mlnew {\mi{root}} {\Top \mlrefine r {\\\displaybreak[3]
&\ \Ldecl {\mi{Unit}} \Bot \Top\\\displaybreak[3]
&\ \mdecl {\mi{unit}} \Top {r.\mi{Unit}}\\\displaybreak[3]
&\ \Ldecl {\mi{Boolean}} \Bot {\Top \mlrefine z {\\\displaybreak[3]
&\ \gap \ldecl {\mi{ifNat}} \arrow {(\arrow {r.\mi{Unit}} {r.\mi{Nat}})} {\arrow {(\arrow {r.\mi{Unit}} {r.\mi{Nat}})} {r.\mi{Nat}}}\\\displaybreak[3]
&\gap}}\\\displaybreak[3]
&\ \mdecl {\mi{false}} {r.\mi{Unit}} {r.\mi{Boolean}}\\\displaybreak[3]
&\ \mdecl {\mi{true}} {r.\mi{Unit}} {r.\mi{Boolean}}\\\displaybreak[3]
&\ \mdecl {\mi{error}} {r.\mi{Unit}} \Bot\\\displaybreak[3]
& \ \ldots\\\displaybreak[3]
&}\mlldefs{\\\displaybreak[3]
&\ \mi{unit}(x) = {\new u {\mi{root}.\mi{Unit}} u}\\\displaybreak[3]
%
&\ \mi{false}(u) =\\\displaybreak[3]
&\ \gap \mlnew {\mi{ff}} {\mi{root}.\mi{Boolean} \mlldefs{\\\displaybreak[3]
&\ \gap\gap {\mi{ifNat}} = \abs t {\arrow {\mi{root}.\mi{Unit}} {\mi{root}.\mi{Nat}}} {\\&\gap\gap\gap\abs e {\arrow {\mi{root}.\mi{Unit}} {\mi{root}.\mi{Nat}}}\\&\gap\gap\gap\app e {\mi{root}.\mi{unit}}}\\\displaybreak[3]
&\ \gap}}{
\ \gap \mi{ff}}\\\displaybreak[3]
%
&\ \mi{true}(u) = \ldots\\\displaybreak[3]
%
&\ \mi{error}(u) = {\app {\mi{root}.\mi{error}} u}\\\displaybreak[3]
&\ \ldots\\\displaybreak[3]
&}}{
\ldots
}
\end{align*}

\subsection{Polymorphic Lists}
Polymorphic lists can be expressed using an abstract type member for
the element type ($\mi{Elem}$). We can instantiate a refinement of the
list package to manipulate lists with a particular element type. We
can choose to make the lists invariant or covariant in the element
type by fixing the lower bound of the $\mi{Elem}$ re-declaration to
the upper bound or to $\Bot$, respectively.
\begin{align*}
&\mlnew {\mi{genLists}} {\Top \mlrefine g {\Ldecl {\mi{ListPackage}} \Bot {\Top \mlrefine p {\\\displaybreak[3]
&\ \Ldecl {\mi{Elem}} \Bot \Top\\
&\ \Ldecl {\mi{ListOfElement}} {g.\mi{List} \refine z {\Ldecl {\mi{Elem}} {p.\mi{Elem}} {p.\mi{Elem}}}} {\\&\gap\gap\gap\gap g.\mi{List} \refine z {\Ldecl {\mi{Elem}} {p.\mi{Elem}} {p.\mi{Elem}}}}\\
&\ \Ldecl {\mi{List}} \Bot {\Top \mlrefine z {\\\displaybreak[3]
&\ \gap \mdecl {\mi{isEmpty}} {\mi{root}.\mi{Unit}} {\mi{root}.\mi{Boolean}}\\
&\ \gap \mdecl {\mi{head}} {\mi{root}.\mi{Unit}} {g.\mi{Elem}}\\
&\ \gap \mdecl {\mi{tail}} {\mi{root}.\mi{Unit}} {g.\mi{ListOfElem}}\\
&\gap}}\\
&\ \mdecl {\mi{nil}} {\mi{root}.\mi{Unit}} {g.\mi{ListOfElem}}\\
&\ \mdecl {\mi{cons}} {g.\mi{Elem}} {g.\mi{ListOfElem}}\\
&}}}}{
\mlnew {\mi{natLists}} {{\mi{genLists}.\mi{ListPackage}} \mlrefine p { \Ldecl {\mi{Elem}} {\mi{Nat}} {\mi{Nat}} }\mlldefs{\\\displaybreak[3]
&\ {\mi{nil}}(u) = \ldots\\
&\ {\mi{cons}}(u) = \ldots\\
&}}}{
\ldots
}
\end{align*}

\section{Counterexamples to Preservation}\label{dot-preservation}

We sketch a proof that the DOT calculus is type-safe using logical
relations in section~\ref{dot-type-safety}. However, we first tried to
prove the calculus type-safe using the standard theorems of
preservation and progress~\cite{soundness,tapl}. Unfortunately, for
the calculus as presented, and any of its variants that we devised,
preservation doesn't hold. In this section, we review some of the most
salient counterexamples to preservation that we found. These
counterexamples have been checked with PLT Redex~\cite{plt_redex}.

\subsection{\texorpdfstring{\textsc{term-$\ni$}}{Term-Mem} Restriction}\label{term_mem}

There are two membership ($t \ni D$) rules: one for when the term $t$
is a path, and one for an arbitrary term $t$. For paths, we can
substitute the self-references in the declarations, but we cannot do
so for arbitrary terms as the resulting types wouldn't be well-formed
syntactically. Hence, the $\textsc{term-$\ni$}$ has the restriction
that self-occurrences are not allowed. Here is a counterexample related
to this restriction.

Let $X$ be a shorthand for the type:
\begin{align*}
\Top & \mlrefine z {\\
&\ \Ldecl {L_a} \Top \Top\\
&\ \ldecl l {z.L_a}\\
}&
\end{align*}

Let $Y$ be a shorthand for the type:
\begin{align*}
\Top & \mlrefine z {\\
&\ \ldecl l \Top\\
}&
\end{align*}

Now, consider the term
\begin{align*}
&\mlnew u {X \ldefs{ l = u }}{
\app {(\abs y {\Top \tfun Y} {\app y u})} {(\abs d \Top {\cast X u})}.l
}
\end{align*}

The term type-checks because the term $t=\app {(\abs y {\Top \tfun Y}
  {\app y u})} {(\abs d \Top {\cast X u})}$ has type
$Y$, so we can apply $\textsc{term-$\ni$}$ for $l$. However, the term
$t$ eventually steps to ${\cast X u}$ which has type $X$,
so we cannot apply $\textsc{term-$\ni$}$ for $l$ because of the
self-reference ($z.L_a$).

\subsection{Expansion Lost}\label{narrowing_exp}

Expansion is not preserved by narrowing. Here, we create two type
selections that are mutually recursive in their upper bounds after
narrowing: $z_0.C_2$ initially expands, but after narrowing, $z_0.C_2$
expands to what $z_0.A_2$ expands to, which expands to what $z_0.A_1$
expands to, which expands to what $z_0.A_2$ expands to, and thus we
have an infinite expansion. Thus, the last new expression initially
type-checks, but after narrowing, it doesn't because the precise
expansion needed by $\textsc{new}$ cannot be inferred.

\begin{align*}
&\mlnew {x_0} {\Top \mlrefine z { \Ldecl {A_1} \Bot {\Top \mlrefine z {\\
&\gap\Ldecl {A_2} \Bot \Top\\
&\gap\Ldecl {A_3} \Bot \Top\\
&\gap\Ldecl {C_2} \Bot {z.A_2}}}}\ldefs{}}{
\mlnew {x_1} {\Top \refine z {\Ldecl {C_1} \Bot {\Top \refine z {\Ldecl {A_1} \Bot {x_0.A_1}}}}\ldefs{}}}{
\mlnew {x_2} {{x_1.C_1} \refine z {\Ldecl {A_1} \Bot {{x_0.A_1} \refine z {\Ldecl {A_2} \Bot {z.A_3}}}}\ldefs{}}}{
\mlnew {x_3} {{x_1.C_1} \refine z {\Ldecl {A_1} \Bot {{x_0.A_1} \refine z {\Ldecl {A_3} \Bot {z.A_2}}}}\ldefs{}}}{
\mlapp {\abs x {x_1.C_1} {(\abs {z_0} {x.A_1 \tand x_3.A_1} {\\&\gap\gap\gap\new z {z_0.C_2} {\app {(\abs x \Top x)} z}})}} {\ x_2}}
\end{align*}

\subsection{Well-Formedness Lost}\label{narrowing_wf}

Even well-formedness is not preserved by narrowing. The trick is that
if the lower bound of a type selection is not $\Bot$, then the
bounds needs to be checked for well-formedness. Here, we create two
type selections that are mutually recursive in their bounds
after narrowing. $y.A$ is initially well-formed, but after narrowing,
it isn't because we run into an infinite derivation trying to prove
the well-formedness of its bounds.

\begin{align*}
&\mlnew v {\Top \refine z {\Ldecl L \Bot {\Top \refine z {\Ldecl A \Bot \Top, \Ldecl B {z.A} {z.A}}}} \ldefs{}}{
\mlapp {(\abs x {\Top \refine z {\Ldecl L \Bot {\Top \refine z {\Ldecl A \Bot \Top, \Ldecl B \Bot \Top}}}} {\\&\gap
\mlnew z {\Top \mlrefine z {\\&\gap\gap\gap
\mdecl l {x.L \tand {\Top \refine z {\Ldecl A {z.B} {z.B}, \Ldecl B \Bot \Top}}} \Top}\mlldefs{\\&\gap\gap\gap
l(y) = \fun a {y.A} \Top a}}{
\gap \cast {\top} z
}})}{\ v}}
\end{align*}

\subsection{Path Equality}\label{patheq}

For preservation, we need to be able to relate path-dependent types
after reduction. Here is a motivating example:

\begin{align*}
&\mlnew b {\Top \mlrefine z {&&\Ldecl X \Top \Top&\\
&&&\ \ldecl l {z.X}&}\ldefs{l = b}}{
\mlnew a {\Top \mlrefine z {\ldecl i {\Top \mlrefine z {&&&\\
&&&\Ldecl X \Bot \Top\\
&&&\ldecl l {z.X}}}&}\ldefs{i = b}}}{
\app {(\abs x \Top x)} {\app {(\abs x {a.i.X} x)} {a.i.l}}}
\end{align*}

$a.i.l$ reduces to $b.l$. $b.l$ has type $b.X$, so we need $b.X <:
a.i.X$. This cannot be established with the current rules: it is not
true in general, but true here because $a.i$ reduces to $b$. Hence,
we need to acknowledge path equality for preservation to hold.

\section{Type Safety via Logical Relations}\label{dot-type-safety}
We sketch a proof of type-safety of the DOT calculus via step-indexed
logical relations~\cite{ahmed04,ahmed06,step_indexed_obj}.

\subsection{Type Safety}
Type-safety states that a well-typed program doesn't get stuck. More
formally: If $\emptyset \ts t \typ T$ and $\reductionl t \emptyset {t'} {s'} *$ then
either $t'$ is a value or $\exists t'', s'', \reduction {t'} {s'}
{t''} {s''}$.

Our strategy is to define a logical relation $\Gamma \ds t : T$, such
that $\Gamma \ts t : T$ implies $\Gamma \ds t : T$ implies
type-safety.

\subsection{Logical Relations}

\subsubsection{Set of Values}
$\relv k \Gamma s T$ defines the set of values that appear to have
type $T$ when taking at most $k$ steps. $\Gamma$ and $s$ must agree:
$\dom(\Gamma) = \dom(s)$ (ordered) and $\forall (x : T) \in \Gamma, x
\in \relv k \Gamma s T$. A variable $y$ belongs to $\relv 0 \Gamma s
T$ simply by being in the store. In addition, it belongs to $\relv k
\Gamma s T$ for $k > 0$, if it defines all method and value labels in
the expansion of $T$ appropriately for $j < k$ steps.
\begin{align*}
&\relv k \Gamma s T = \{ y \ |\\
&\gap \Gamma \ts T \wfe \andl\\
&\gap y \in \dom(s) \andl\\
&\gap \forall j < k,\\
&\gap y \mapsto T_c \ldefs{\seq{l = v}\;\seq{m(x)=t}} \in s,\\
&\gap \Gamma \ts T \expand_y \seq{D},\\
&\gap (\forall m_i : S \rightarrow U \in \seq{D},\\
&\gap\gap t_i \in \rele j {\Gamma \envplus{x : S}} s U)\\
&\gap \andl\\
&\gap (\forall l_i : V \in \seq{D},\\
&\gap\gap v_i \in \relv j \Gamma s V)\\
&\}
\end{align*}

\subsubsection{Set of Terms}
$\rele k \Gamma s T$ defines the set of terms that appear to have type
$T$ when taking at most $k$ steps. $s$ must agree with a {\it prefix}
of $\Gamma$, so $\Gamma$ can additionally contain variables not in
$s$. This is needed for checking methods in $\mathcal{V}$ above, and
for relating open terms. If $k > 0$, $\mathcal{E}$ extends $\Gamma$
and $s$ so that they agree. It then states that if it can reduce $t$
in the extended store to an irreducible term in $j < k$ steps, then
this term must be in a corresponding $\mathcal{V}$ set.
\begin{align*}
&\rele k \Gamma s T = \{ t \ |\\
&\gap \Gamma \ts T \wfe \andl\\
&\gap k = 0 \orl (\forall j \leq k,\\
&\gap\gap \forall (\Gamma'; s') \in \rels k \Gamma s ,\\
&\gap\gap \reductionl t {s'} {t'} {s''} j \andl\\
&\gap\gap \irred {t'} {s''} \impliesl\\
&\gap\gap \forall \Gamma'' \in \relg k {s''} {\Gamma'} ,\\
&\gap\gap {t'} \in \relv {k-j-1} {\Gamma''} {s''} T)\\
&\}
\end{align*}

\subsubsection{Extending the environment and the store}
$\rels k \Gamma s$ for $k > 0$ defines the set of completed
environment and stores that agree on $k-1$ steps, and that extend
$\Gamma$ and $s$. $s$ must agree with a {\it prefix} of $\Gamma$.
Both $\Gamma$ and $s$ are ordered maps. For $s$, $s'$ extends $s$ if
$s$ is a prefix of $s'$. For $\Gamma$, $\Gamma'$ extends $\Gamma$ if
we get back $\Gamma$ by keeping only the elements of $\Gamma'$ that
belong to $\Gamma$. Furthermore, a prefix of $\Gamma'$ agrees with
$s$.
\begin{align*}
&\rels k \Gamma s = \{\\
&\ (\seq{x : T}^m, \seq{x_{ij} : T_{ij}}^{m \leq i < n; 0 \leq j < i_n}; s, \seq{x_{ij} \mapsto c_{ij}}^{m \leq i < n; 0 \leq j < i_n}) |\\
&\ s = \seq{x \mapsto c}^m \andl \Gamma = \seq{x : T}^n \andl \\
&\ m\leq n \andl \forall i, m \leq i < n, \forall {i_n}, j, 0 \leq j < i_n,\\
&\ \forall T_{ij}, c_{ij}, T_{i{(i_n-1)}} = T_i, \forall n' \leq n, i_n' \leq i_n,\\
&\ c_{ij} \in \relv {k-1} {\seq{x : T}^m, \seq{x_{ij} : T_{ij}}^{m \leq i' < n; 0 \leq j < i_n'}} {s, \seq{x_{ij} \mapsto c_{ij}}^{m \leq i < n'; 0 \leq j < i_n'}} {T_{ij}}\\
&\}
\end{align*}

\subsubsection{Completing the environment to agree with the store}
$\relg k s \Gamma$ defines a singleton set of a completed environment
that agrees with a store $s$ by simply copying the constructor type
from the store for each missing variable.
\begin{align*}
&\relg k s \Gamma = \{ \Gamma, \seq{x_i : T_{c_i}}^{m \leq i < n} \ |\\
&\gap \Gamma = \seq{x : T}^m \andl s = \seq{x \mapsto c}^n \andl\\
&\}
\end{align*}

\subsubsection{$\Gamma \ds t : T$}

$\Gamma \ds t : T$ is simply defined as $t \in \rele k \Gamma \emptyset T,
\forall k$.

\subsection{Statements and Proofs}

\subsubsection{Fundamental Theorem}

The fundamental theorem is the implication from $\Gamma \ts t : T$ to
$\Gamma \ds t : T$. Type safety is a straightforward corollary of this
theorem.

\begin{myproof}
The proof is on induction on the derivation of $\Gamma \ts t : T$. For
each case, we need to show $t \in \rele k \Gamma \emptyset T, \forall k$. The
non-trivial case is to need to show ${t'} \in \relv {k-j-1} {\Gamma''}
{s'} T$ by supposing $k > 0$, $(\Gamma'; s) \in \rels k \Gamma \emptyset$, and, for some
$j \leq k$, $\reductionl {t} s {t'} {s'} j \andl \irred {t'}
{s'} \andl {\Gamma''} \in \relg \Gamma k {s'} {\Gamma'}$.

\ \\

\Case \textsc{var}: $\Gamma \ts x : T$ knowing $(x : T) \in
\Gamma$. $x \in \relv {k-1} {\Gamma'} s T$ follows from the definition of
$\rels k \Gamma \emptyset$.

\ \\

\Case \textsc{sel}: $\Gamma \ts t_1.l_i : T$ knowing $\Gamma \ts t_1 :
T_1$, $\Gamma \ts T_1 \expand_z \seq{D}$, $\ldecl {l_i} {V_i} \in
\seq{D}$ and knowing either that $t_1 = p_1 \andl T = \subst p z
    {V_i}$ or that $z \not\in \fn(V_i) \andl T = V_i$.

By operational semantics and induction hypothesis, $\reductionl {t_1}
s {t_1'} {s'} {j-1}$ and $\irred {t_1'} {s'}$ and ${t_1'} \in \relv
{k-j+1-1} {\Gamma'} {s'} {T_1}$.

By operational semantics and the above, $\reductionl {t_1'.l_i} {s'}
{t'} {s'} 1$, and we can conclude $t' \in \relv {k-j-1} {\Gamma''} {s'}
T$ from the clause for value labels of ${t_1'} \in \relv {k-j}
{\Gamma''} {s'} {T_1}$.

\ \\

\Case \textsc{msel}: $\Gamma \ts t_1.m_i(t_2) : T$ knowing $\Gamma \ts
t_1 : T_1$, $\Gamma \ts t_2 : T_2$, $\Gamma \ts T_1 \expand_z
\seq{D}$, $\mdecl {m_i} {S_i} {U_i} \in \seq{D}$ and knowing either
that $t_1 = p_1 \andl S = \subst p z {S_i} \andl T = \subst p z {U_i}$
or that $z \not\in \fn(S_i) \andl z \not\in \fn(U_i) \andl S = S_i \andl
T = U_i$, and knowing that $\Gamma \ts T_2 \sub S$.

By operational semantics and induction hypotheses, $\reductionl {t_1}
s {t_1'} {s_1} {j_1}$ and $\irred {t_1'} {s_1}$ and $\reductionl {t_2}
{s} {t_2'} {s_2} {j_2}$ and $\irred {t_2'} {s_2}$ and ${t_1'} \in
\relv {k-j_1-1} {\Gamma_1} {s_1} {T_1}$ and ${t_2'} \in \relv
{k-j_2-1} {\Gamma_2} {s_2} {T_2}$.

Because $t_2$ reduces to a value $t_2'$ starting in store $s$, it
should also reduce to a value $v_2$ in the same number of steps
starting in store $s_1$, since $s_1$ extends $s$. So let $\reductionl
{t_2} {s_1} {v_2} {s_{12}} {j_2}$ with $v_2 \in \relv {k-j_2-1}
{\Gamma_{12}} {s_{12}} {T_2}$.

By the above and operational semantics, $\reductionl {t_1'.m_i(v_2)}
{s_{12}} {\subst {v_2} {x_i} {t_i}} {s_{12}} 1$.

By the substitution lemma, ${\subst {v_2} {x_i} {t_i}} \in \rele
{k-\max(j_1,j_2)-1} {\Gamma_{12}} {s_{12}} {T}$. Supposing,
$\reductionl {\subst {v_2} {x_i} {t_i}} {s_{12}} {t'} {s'} {j_3}$,
with $j_1 + j_2 + j_3 + 1 = j$, this completes the case, by
monotonicity of $\mathcal{V}$.

\ \\

\Case \textsc{new}: $\Gamma \ts \new y c {t_b} : T$ knowing ...

By operational semantics, $\reductionl {\new y c {t_b}} s {t_b} {s_b} 1$
where $s_b = s \envplus{y \mapsto c}$. So $\reductionl {t_b} {s_b} {t'}
{s'} {j-1}$.

By induction hypotheses, $y \in \relv k {\Gamma_b} {s_b} {T_c}$ and $t_b \in
\rele k {\Gamma_b} {s_b} {T}$.

Result follows by monotonicity of $\mathcal{V}$.

\end{myproof}

\subsubsection{Substitution Lemma}

The substitution lemma states that if (1) $v \in \relv {k_2}
{\Gamma_{12}} {s_{12}} {T_2}$ and (2) $t \in \rele {k_1} {\Gamma_1
  \envplus{{x} : S}} {s_1} T$ and (3) $\Gamma \ts {T_2} \sub S$ with
(4) $x \not\in \fn(T)$ and $\Gamma_1$ extends $\Gamma$ and
$\Gamma_{12}$ included in $\Gamma_1$ and $s_{12}$ extends $s_1$ and
$\Gamma_1$ agrees with $s_1$ and $\Gamma_{12}$ agrees with $s_{12}$
and a prefix of $\Gamma_{12}$ agrees with $s_1$, then $\subst {v} {x}
{t} \in \rele {\min(k_1,k_2)} {\Gamma_{12}} {s_{12}} T$.

\begin{proofsketch}
By (1) and (3), it should hold that (5) $v \in \relv {k_2}
{\Gamma_{12}} {s_{12}} {S}$ by the subset semantics lemma.  Since (2)
holds, it should also hold that ${t} \in \rele {\min(k_1,k_2)}
{\Gamma_{12} \envplus{{x} : S}} {s_{12}} T$ by the extended
monotonicity lemma. Then, we can instantiate $x$ in the complete store
to map to what $v$ maps to. This should be fine by (5) and
monotonicity. Thus, ${t} \in \rele {\min(k_1,k_2)} {\Gamma_{12}
  \envplus{{x} : S}} {s_{12} \envplus{{x} \mapsto s_{12}(v)}}
T$. Thanks to (4), we don't actually need $x$ to be held abstract in
the environment, because it won't occur in $T$ or its expansion, so we
can use the actual type constructor of $v$, say $T_{v}$, instead of
$S$ for $x$: ${t} \in \rele {\min(k_1,k_2)} {\Gamma_{12} \envplus{{x}
    : {T_{v_2}}}} {s_{12} \envplus{{x} \mapsto s_{12}(v)}} T$. This
implies what needs to be shown.

\end{proofsketch}

\subsubsection{Subset Semantics Lemma}

The subset semantics lemma states that if (1) $v \in \relv k \Gamma s S$
and (2) $\Gamma \ts S <: U$, then $v \in \relv k \Gamma s U$.

\begin{proofsketch}
Because $S$ is a subtype of $U$, it should hold that the expansion of
$S$ subsumes the expansion of $U$. Therefore, for $v \in \relv k
\Gamma s U$, we have fewer and ``simpler'' definitions to check than
for $v \in \relv k \Gamma s S$. One worry is that for methods, because
the parameter type is contravariant, $U$ could require a more precise
type than $S$. This shouldn't lead to narrowing issues, because if the
parameter type becomes ill-typed, then the check is vacuously true.
\end{proofsketch}

\subsubsection{Extended Monotonicity Lemma}

The extended monotonicity lemma states that if $t \in \rele k {\Gamma
  \envplus{x : S}} s T$ then $t \in \rele j {\Gamma' \envplus{x : S}}
{s'} T$ for $j \leq k$, $\Gamma'$ extends $\Gamma$, $s'$
extends $s$, and $\Gamma$ agrees with $s$ and a prefix of $\Gamma'$
agrees with $s$.

\begin{proofsketch}
For the monotonicity with regards to the step-index, this follows
directly from the definitions of $\mathcal{E}$ and $\mathcal{V}$. For the
environment and the store, this follows from the definition of $\rels
k {\Gamma \envplus{x : S}} s$. To extend the environment and the store
for $x : S$, we can append as much as we want to $\Gamma$ and $s$, to
get $\Gamma'$ and $s'$, and then ignore the last element which is for
$x : S$.
\end{proofsketch}

\section{Discussion}\label{discussion}

\subsection{Variants of the DOT Calculus}\label{dot-variants}

\subsubsection{Why limit the calculus to $\wfe$-types?}

Currently, the proof of type-safety via logical relations
fundamentally relies on types having an expansion. However, this was
not our original motivation for limiting the calculus to $\wfe$-types.

Originally, subtyping was not regular wrt $\wfe$. Roughly,
all the $\wfe$ preconditions in subtyping were dropped. In this
broader calculus, subtyping transitivity doesn't hold, because of the
rule (\textsc{$\sub$-rfn}) which requires expansion of the left type.

The problem is deep, as attested by this elaborate counterexample
that is not so easily patched, and directly leads to a counterexample
to preservation.

Consider an environment where $u$ is bound to:
\begin{align*}
\Top & \mlrefine u {\\
&\ \Ldecl {\mathit{Bad}} {\Bot} {u.\mathit{Bad}}\\
&\ \Ldecl {\mathit{Good}} {\Top \refine z {\Ldecl L \Bot \Top}} {\Top \refine z {\Ldecl L \Bot \Top}}\\
&\ \Ldecl {\mathit{Lower}} {u.\mathit{Bad} \tand u.\mathit{Good}} {u.\mathit{Good}}\\
&\ \Ldecl {\mathit{Upper}} {u.\mathit{Good}} {u.\mathit{Bad} \tor u.\mathit{Good}}\\
&\ \Ldecl X {u.\mathit{Lower}} {u.\mathit{Upper}}\\
}&
\end{align*}

Now, consider the types $S$, $T$, $U$ defined in terms of $u$:
\begin{align*}
S &= u.\mathit{Bad} \tand u.\mathit{Good}\\
T &= u.\mathit{Lower}\\
U &= u.X \refine z {\Ldecl L \Bot \Top}
\end{align*}
We have $S \sub T$ and $T \sub U$, but we cannot derive $S \sub U$ because
$S$ doesn't expand.

Note that $u$ is realizable, since each lower bound is a subtype of
its upper bound. So it is straightforward to turn this counterexample
to subtyping transitivity into a counterexample to preservation:
\begin{align*}
&\mlnew u \ldots {
\ \mlapp{\abs x \Top x}{
\ \ \mlapp{\abs f {S \tfun U} f}{
\ \ \ \mlapp{\abs f {S \tfun T} f}{
\ \ \ \ \mlapp{\abs f {S \tfun S} f}{
\ \ \  \ \ \abs x S x}}}}}
\end{align*}

The idea is to start with a function from $S \tfun S$ and cast it
successively to $S \tfun T$ then $S \tfun U$. To typecheck the
expression initially, we need to check $S \sub T$ and $T \sub
U$. After some reduction steps, the first few casts vanish, and the
reduced expression casts directly from $S \tfun S$ to $S \tfun U$, so
we need to check $S \sub U$.

\subsubsection{Why not include the lambda-calculus instead of methods?}

Originally, the DOT calculus included the lambda-calculus, and
explicit methods were not needed since they could be represented by a
value label with a function type. However, the expansion of the
function type was defined to be the empty set of declarations (like
for $\Top$), which caused a real breach of type-safety.

A concrete object could be a subtype of a function type without a
function ever being defined. Consider:
\begin{align*}
&\mlnew u {\Top \refine z {\Ldecl C {\Top \tfun \Top} {\Top \tfun \Top}} \ldefs{}} {
\mlnew f {u.C \ldefs{}} {
\ldots
}}
\end{align*}

Now, $f$ was a subtype of $\Top \tfun \Top$, but $\app f {(\abs x \Top
  x)}$ was stuck (and, rightfully, didn't typecheck). But we could use
narrowing to create a counterexample to type safety: $\app {(\abs g
  {\Top \tfun \Top} {\app g (\abs x \Top x)})} f$.

Because of this complication, we decided to drop the lambda-calculus
from DOT, and instead introduce methods with one parameter. Like in
Scala, functions are then just sugar for objects with a special
method.

An alternative design would have been to change the expansion of the
function type to have a declaration for a special marker value label.

\subsubsection{Why not patch the DOT calculus for preservation to hold?}

We tried! However, the resulting calculi were not elegant, and
furthermore, we still found issues with preservation. Below, we give a
brief summary of one failed attempt to patch the calculus for
preservation to hold.

Because many of the counterexamples to preservation are related to
narrowing, we tried to make widening an explicit operation and change
rules with implicit relaxations (\textsc{msel} and \textsc{new}) to be
strict. From a typing perspective, the change was straightforward, but
reduction became more complicated and dependent on typing because the
type information in widenings needed to be propagated correctly.

We added path equality provisions in the subtyping rules, in the same
spirit as the Tribe calculus~\cite{tribe}.

Unfortunately, these two patches interacted badly, and we were left
with a disturbing counterexample to type safety.

\begin{align*}
&\mlnew a {\Top \refine z {\Ldecl C \Bot {\Top \refine z {\Ldecl D \Bot {z.X}, \Ldecl X \Bot \Top}}}}{
\mlnew b {{a.C} \refine z {\Ldecl X \Bot \Bot}}}{
\mlnew c {a.C}}{
\mlnew d {{\cast {a.C} b}.D}}{
\app {(\abs x \Bot {x.\mathit{foo}})} d
}
\end{align*}

Notice that $d$ has type $\Bot$ if the cast on $b$ is ignored. This
example didn't typecheck initially because the path-equality
provisions only applied when objects are in the store, so the
application was not well-typed. But if we started preservation in a store
which had $a$, $b$, $c$ and $d$ then the application type-checked,
because, through one of the path-equality provision, we could find that
the type of $d$ was a subtype of $\Bot$. Now, of course, when we got to
$d.\mathit{foo}$, reduction failed.

\section{Conclusion}\label{conclusion}

We have presented DOT, a calculus aimed as a new foundation of Scala
and languages like it. DOT features path-dependent types, refinement
types, and abstract type members.

Proving the DOT calculus type-safe has been an interesting
adventure. We have shown that DOT does not satisfy preservation (also
known as subject-reduction). However, the standard theorems of
preservation and progress are just one way to prove type safety. We
have sketched a plausible proof of type safety by using the powerful
method of logical relations.

%\appendix
%\section{Appendix Title}

\acks

We thank Amal Ahmed for many discussions and insights about applying
logical relations to DOT. We thank Donna Malayeri and Geoffrey
Washburn for preliminary work on DOT.

\bibliographystyle{abbrvnat}
\bibliography{dot}

\end{document}
