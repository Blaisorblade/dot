%-----------------------------------------------------------------------------
%
%               Template for sigplanconf LaTeX Class
%
% Name:         sigplanconf-template.tex
%
% Purpose:      A template for sigplanconf.cls, which is a LaTeX 2e class
%               file for SIGPLAN conference proceedings.
%
% Guide:        Refer to "Author's Guide to the ACM SIGPLAN Class,"
%               sigplanconf-guide.pdf
%
% Author:       Paul C. Anagnostopoulos
%               Windfall Software
%               978 371-2316
%               paul@windfall.com
%
% Created:      15 February 2005
%
%-----------------------------------------------------------------------------


\documentclass[9pt]{sigplanconf}

% The following \documentclass options may be useful:
%
% 10pt          To set in 10-point type instead of 9-point.
% 11pt          To set in 11-point type instead of 9-point.
% authoryear    To obtain author/year citation style instead of numeric.

\usepackage{amsmath}

\usepackage{fleqn}
\usepackage{listings}
\usepackage{math}
\usepackage{amsmath}
\usepackage{latexsym}
\usepackage{bcprules}
\usepackage[scaled=0.848971]{luximono} % This is for 11 pt Default font
\usepackage[T1]{fontenc}

% Prooftree formatting
\usepackage{prooftree}

\usepackage{multicol}
\usepackage{framed}

%\usepackage{float}
%\floatstyle{boxed} 
%\restylefloat{figure}

% support for generating PDF files
%\newif\ifpdf
%    \ifx\pdfoutput\undefined
%    \pdffalse
%\else
%    \pdftrue
%    \pdfoutput=1
%\fi

%versions
% Use dependent function types
\newif\ifdep\depfalse

\lstset{
  literate=
  {=>}{$\Rightarrow\;$}{2}
  {<:}{$<:\;$}{1}
}

\lstdefinelanguage{scala}{% 
       morekeywords={% 
                try, catch, throw, private, public, protected, import, package, implicit, final, package, trait, type, class, val, def, var, if, for, this, else, extends, with, while, new, abstract, object, case, match, sealed,override},
         sensitive=t, % 
   morecomment=[s]{/*}{*/},morecomment=[l]{\//},% 
   mathescape,
%   escapeinside={/*\%}{*/},%
   rangeprefix= /*< ,rangesuffix= >*/,%
   morestring=[d]{"}% 
 }
 
\lstset{breaklines=true,language=scala} 

\def\code{\lstinline}  % shorter version so you can write \code|String[Foo]|
                       % -- \def must be in same file as uses for this to
                       % work...
\newcommand{\lstref}[1]{Listing~\ref{#1}}
\newcommand{\Lstref}[1]{Listing~\ref{#1}} % only capitalise at beginning of sentence?
\newcommand{\secref}[1]{Section~\ref{#1}}
\newcommand{\Secref}[1]{Section~\ref{#1}} % only capitalise at beginning of sentence?



% \lstset{basicstyle=\footnotesize\ttfamily, breaklines=true, language=scala, tabsize=2, columns=fixed, mathescape=false,includerangemarker=false}
% thank you, Burak 
% (lstset tweaking stolen from
% http://lampsvn.epfl.ch/svn-repos/scala/scala/branches/typestate/docs/tstate-report/datasway.tex)
\lstset{
    xleftmargin=2em,%
    framesep=5pt,%
    frame=none,%
    captionpos=b,%
    fontadjust=true,%
    columns=[c]fixed,%
    keepspaces=false,%
    basewidth={0.56em, 0.52em},%
    tabsize=2,%
    basicstyle=\small\tt,% \small\tt
    commentstyle=\textit,%
    keywordstyle=\bfseries,%
    escapechar=\%,%
}

%% set latex/pdflatex specific stuff
%\ifpdf
    \usepackage[pdftex,
                hyperindex,
                plainpages=false,
                breaklinks,
                colorlinks,
                citecolor=black,
                filecolor=black,
                linkcolor=black,
                pagecolor=black,
                urlcolor=black]{hyperref}
    \usepackage[pdftex]{graphicx}
    \DeclareGraphicsExtensions{.jpg,.pdf}
    \pdfcatalog {
        /PageMode (/UseNone)
    }
    \usepackage{thumbpdf}
    \usepackage[pdftex]{color}
%\else
%    \usepackage[ps2pdf]{hyperref}
%    \usepackage{graphicx}
%    \DeclareGraphicsExtensions{.eps,.jpg}
%    \usepackage{color}
%\fi

%\setlength{\parindent}{0pt}
%\setlength{\parskip}{5pt}

% verbfilter stuff
\newcommand{\prog}[1]{{\sl #1}}
\newenvironment{program}[1][10.5]
  {\fontsize{#1}{13.6}\tt\begin{tabbing}\hspace*{0.5\parindent}\=\+\kill}
  {\end{tabbing}\noindent}
\newcommand{\blockcomment}[1]{{\color{grayPoint3}#1}}
\newcommand{\linecomment}{\color{grayPoint3}}
\newcommand{\grey}{\color{grey}}

%\newenvironment{program}{\ \ \ \ \begin{minipage}{\textwidth}\renewcommand{\baselinestretch}{1.0}\sl\begin{tabbing}}{\end{tabbing}\end{minipage}}
\newcommand{\vem}{\bfseries}
\newcommand{\quotedstring}[1]{{#1}}
\newcommand{\typename}[1]{{#1}}
\newcommand{\literal}[1]{{#1}}

% comments and notes
\newcommand{\comment}[1]{}
%\newcommand{\note}[1]{{\bf $\clubsuit$ #1 $\spadesuit$}}

% figures
\newcommand{\figurebox}[1]
        {\fbox{\begin{minipage}{\textwidth} #1 \medskip\end{minipage}}}
%        {\fbox{\begin{minipage}{\textwidth}\begin{center} #1 \end{center}\medskip\end{minipage}}}
\newcommand{\boxfig}[3]
        {\begin{figure*}\figurebox{#3\caption{\label{fig:#1}#2}}\end{figure*}}
\newcommand{\figref}[1]
        {Figure~\ref{fig:#1}}

% typing rules (not used here)
\newcommand{\ttag}[1]{\mbox{\textsc{\small(#1)}}}
\newcommand{\infer}[3]{\mbox{#1 }\ba{c} #2 \\ \hline #3 \ea}
\newcommand{\irule}[2]{{\renewcommand{\arraystretch}{1.2}\ba{c} #1 
                        \\ \hline #2 \ea}}
\newlength{\trulemargin}
\newlength{\trulewidth}
\newlength{\srulewidth}
\setlength{\trulemargin}{0.80cm}
\setlength{\trulewidth}{40.0mm}
\setlength{\srulewidth}{3.0cm}
\newenvironment{trules}{$\vspace{0.5em}\ba{p{\trulemargin}@{~}p{\trulewidth}@{~}p{\trulemargin}}}{\ea$}
\newenvironment{srules}{$\vspace{0.5em}\ba{p{\trulemargin}@{~}p{\srulewidth}}}{\ea$}
\newcommand{\laxiom}[2]{\ttag{#1} & $ #2 \hfill\ }
\newcommand{\raxiom}[2]{\hfill #2 $& \hfill \ttag{#1}}
\newcommand{\caxiom}[2]{\ttag{#1} & $\hfill #2 \hfill $& \ }
\newcommand{\lrule}[3]{\laxiom{#1}{\irule{#2}{#3}}}
\newcommand{\rrule}[3]{\raxiom{#1}{\irule{#2}{#3}}}
\newcommand{\crule}[3]{\caxiom{#1}{\irule{#2}{#3}}}
\newcommand{\lsrule}[3]{\lsaxiom{#1}{\irule{#2}{#3}}}
\newcommand{\rsrule}[3]{\rsaxiom{#1}{\irule{#2}{#3}}}
\newcommand{\nl}{\end{trules}\\[0.5em] \begin{trules}}
\newcommand{\snl}{\end{srules}\\[0.5em] \begin{srules}}

% commas and semicolons
\newcommand{\comma}{,\,}
\newcommand{\commadots}{\comma \ldots \comma}
\newcommand{\semi}{;\mbox{;};}
\newcommand{\semidots}{\semi \ldots \semi}

% spacing
\newcommand{\gap}{\quad\quad}
\newcommand{\biggap}{\quad\quad\quad}
\newcommand{\nextline}{\\ \\}
\newcommand{\htabwidth}{0.5cm}
\newcommand{\tabwidth}{1cm}
\newcommand{\htab}{\hspace{\htabwidth}}
\newcommand{\tab}{\hspace{\tabwidth}}
\newcommand{\linesep}{\ \hrulefill \ \smallskip}

% math stuff
\newenvironment{myproof}{{\em Proof:}}{$\Box$}
\newenvironment{proofsketch}{{\em Proof Sketch:}}{$\Box$}
\newcommand{\Case}{{\em Case\ }}

% make ; a delimiter in math mode
% \mathcode`\;="8000 % Makes ; active in math mode
% {\catcode`\;=\active \gdef;{\;}}
% \mathchardef\semicolon="003B

% reserved words
\newcommand{\mathem}{\bf}

% brackets
\newcommand{\set}[1]{\{#1\}}
\newcommand{\sbs}[1]{\lquote #1 \rquote}

% arrays
\newcommand{\ba}{\begin{array}}
\newcommand{\ea}{\end{array}}
\newcommand{\bda}{\[\ba}
\newcommand{\eda}{\ea\]}
\newcommand{\ei}{\end{array}}
\newcommand{\bcases}{\left\{\begin{array}{ll}}
\newcommand{\ecases}{\end{array}\right.}

% \cal ids
\renewcommand{\AA}{{\cal A}}
\newcommand{\BB}{{\cal B}}
\newcommand{\CC}{{\cal C}}
\newcommand{\DD}{{\cal D}}
\newcommand{\EE}{{\cal E}}
\newcommand{\FF}{{\cal F}}
\newcommand{\GG}{{\cal G}}
\newcommand{\HH}{{\cal H}}
\newcommand{\II}{{\cal I}}
\newcommand{\JJ}{{\cal J}}
\newcommand{\KK}{{\cal K}}
\newcommand{\LL}{{\cal L}}
\newcommand{\MM}{{\cal M}}
\newcommand{\NN}{{\cal N}}
\newcommand{\OO}{{\cal O}}
\newcommand{\PP}{{\cal P}}
\newcommand{\QQ}{{\cal Q}}
\newcommand{\RR}{{\cal R}}
\newcommand{\TT}{{\cal T}}
\newcommand{\UU}{{\cal U}}
\newcommand{\VV}{{\cal V}}
\newcommand{\WW}{{\cal W}}
\newcommand{\XX}{{\cal X}}
\newcommand{\YY}{{\cal Y}}
\newcommand{\ZZ}{{\cal Z}}

% misc symbols
\newcommand{\dhd}{\!\!\!\!\!\rightarrow}
\newcommand{\Dhd}{\!\!\!\!\!\Rightarrow}
\newcommand{\ts}{\,\vdash\,}
\newcommand{\la}{\langle}
\newcommand{\ra}{\rangle}
\newcommand{\eg}{{\em e.g.}}

% misc identifiers
\newcommand{\dom}{\mbox{\sl dom}}
\newcommand{\fn}{\mbox{\sl fn}}
\newcommand{\bn}{\mbox{\sl bn}}
\newcommand{\sig}{\mbox{\sl sig}}
\newcommand{\IF}{\mbox{\mathem if}}
\newcommand{\OTHERWISE}{\mbox{\mathem otherwise}}
\newcommand{\expand}{\prec}
\newcommand{\weakexpand}{\prec^W}
\newcommand{\spcomma}{~,~}

%\newcommand{\inst}{\mbox{\mathem inst}}
\newcommand{\trans}[1]{\la\!\la#1\ra\!\ra}
\newcommand{\remark}[1]{{\bf $\clubsuit$ #1 $\spadesuit$}}
\newcommand{\todo}[1]{\remark{to do: #1}}
%\newcommand{\J}{\justifies}
%\newcommand{\U}{\using}

% names
\newcommand{\Scala}{\mbox{\textsc{Scala}}}
\newcommand{\Java}{\mbox{\textsc{Java}}}

%\renewcommand\textfraction{.05}
%\renewcommand\floatpagefraction{.9}
%\renewcommand\topfraction{.8}

%%%%%%%%%%%%%%%%%%%%%%%%%%%%%%%%%%%%%%%
%   Language abstraction commands     %
%%%%%%%%%%%%%%%%%%%%%%%%%%%%%%%%%%%%%%%

%% Relations
% Subtype 
\newcommand{\sub}{<:}
% Type assignment
\newcommand{\typ}{:}
% reduction
\newcommand{\reduces}{\;\rightarrow\;}
% well-formedness
\newcommand{\wf}{\;\mbox{\textbf{wf}}}
\newcommand{\wfe}{\;\mbox{\textbf{wfe}}}

%% Operators
% Type selection
\newcommand{\tsel}{\#}
% Function type
\newcommand{\tfun}{\rightarrow}
\newcommand{\dfun}[3]{(#1\!:\!#2) \Rightarrow #3}
% Conjunction
\newcommand{\tand}{\wedge}
% Disjunction
\newcommand{\tor}{\vee}
% Singleton type suffix
\newcommand{\sing}{.\textbf{type}}

%% Syntax
% Header for typing rules
\newcommand{\judgement}[2]{{\bf #1} \hfill #2}
% Refinement
\newcommand{\refine}[2]{\left\{#1 \Rightarrow #2 \right\}}
% Field definitions
\newcommand{\ldefs}[1]{\left\{#1\right\}}
% Member sequences
\newcommand{\seq}[1]{\overline{#1}}
% Lambda
\newcommand{\dabs}[3]{(#1\!:\!#2)\Rightarrow #3}
\newcommand{\abs}[3]{\lambda #1\!:\!#2.#3}
% Application
\newcommand{\app}[2]{#1\;#2}
% Method Application
\newcommand{\mapp}[3]{#1.#2(#3)}
% Substitution
\newcommand{\subst}[3]{[#1/#2]#3}
% Object creation
\newcommand{\new}[3]{\textbf{val }#1 = \textbf{new }#2 ;\; #3}
%\renewcommand{\new}[3]{#1 \leftarrow #2 \,\textbf{in}\, #3}
% Field declaration
\newcommand{\Ldecl}[3]{#1 : #2..#3}%{#1 \operatorname{>:} #2 \operatorname{<:} #3}
\newcommand{\ldecl}[2]{#1 : #2}
\newcommand{\mdecl}[3]{#1 : #2 \tfun #3}
% Top and Bottom
\newcommand{\Top}{\top}%{\textbf{Top}}
\newcommand{\Bot}{\bot}%\textbf{Bot}}
% Environment extension
%\newcommand{\envplus}[1]{\uplus \{ #1 \}}
\newcommand{\envplus}[1]{, #1}
% Reduction
\newcommand{\reduction}[4]{#1 \operatorname{|} #2 \reduces #3 \operatorname{|} #4}

\newcommand{\lindent}{\hspace{-4mm}}

\begin{document}

\conferenceinfo{FOOL '12}{October 22, 2012, Tucson, AZ, USA.} 
\copyrightyear{2012} 
\copyrightdata{[to be supplied]} 

\title{Dependent Object Types}
\subtitle{A foundations for Scala's type system} % TODO: ask

\authorinfo{Nada Amin \and Adriaan Moors \and Martin Odersky}
           {EPFL}
           {first.last@epfl.ch}

\maketitle

\begin{abstract}
We propose a new type-theoretic foundation of Scala and languages like
it: the Dependent Object Types calculus (DOT). DOT models Scala's
path-dependent types and abstract type members, as well as its mixture
of nominal and structural typing through the use of refinement
types. It makes no attempt to model inheritance or mixing
composition. The calculus does not model what's currently in Scala: it
is more normative than descriptive.

We show that DOT and its patched-up variants are not syntactically
sound, by exhibiting counterexamples to preservation. Nevertheless, we
prove type-safety of the calculus via step-indexed logical relations.
\end{abstract}

\category{D.3.3}{Language Constructs and Features}{Abstract data types, Classes and objects, polymorphism}
\category{D.3.1}{Formal Definitions and Theory}{Syntax, Semantics}
\category{F.3.1}{Specifying and Verifying and Reasoning about Programs}{}
\category{F.3.3}{Studies of Program Constructs}{Object-oriented constructs, type structure}
\category{F.3.2}{Semantics or Programming Languages}{Operational semantics}

\terms
Languages, Theory, Verification

\keywords
calculus, objects, dependent types, step-indexed logical relations

\section{Introduction}

This paper presents a proposal for a new type-theoretic foundation of
Scala and languages like it. The properties we are interested in
modelling are Scala's path-dependent types and abstract type members,
as well as its mixture of nominal and structural typing through the
use of refinement types. Compared to previous approaches (nuObj, FS),
we make no attempt to model inheritance or mixin composition. Indeed
we will argue that such concepts are better modelled in a different
setting.

The calculus does not precisely describe what's currently in Scala. It
is more normative than descriptive. The main point of deviation
concerns the difference between Scala's compound type formation using
{\bf with} and classical type intersection, as it is modelled in the
calculus. Scala, and the previous calculi attempting to model it
conflates the concepts of compound types (which inherit the members of
several parent types) and mixin composition (which build classes from
other classes and traits). At first glance, this offers an economy of
concepts. However, it is problematic because mixin composition and
intersection types have quite different properties. In the case of
several inherited members with the same name, mixin composition has to
pick one which overrides the others. It uses for that the concept of
linearization of a trait hierarchy. Typically, given two independent
traits $T_1$ and $T_2$ with a common method $m$, the mixin composition
\code@$T_1$ with $T_2$@ would pick the $m$ in $T_2$, whereas the member in
$T_1$ would be available via a super-call. All this makes sense from
an implementation standpoint. From a typing standpoint it is more
awkward, because it breaks commutativity and with it several
monotonicity properties.

In the present calculus, we replace Scala's compound types by
classical intersection types, which are commutative. We also
complement this by classical union types. Intersections and unions
form a lattice wrt subtyping. This addresses another problematic
feature of Scala: In Scala's current type system, least upper bounds
and greatest lower bounds do not always exist. Here is an example:
Given two traits
\begin{lstlisting}
  trait A { type T <: A }
  trait B { type T <: B }
\end{lstlisting}
The greatest lower bound of \code@A@ and \code@B@ is approximated by the
infinite sequence
\begin{lstlisting}
  A with B { type T <: A with B { type T <: A with B {
    type T < ...
  }}}
\end{lstlisting}
The limit of this sequence does not exist as a type in Scala.

This is problematic because glbs and lubs play a central role in
Scala's type inference. The absence of universal glbs and lubs makes
type inference more brittle and more unpredictable.

\subsection*{Why No Inheritance?}

In the calculus we made the deliberate choice not to model any form of
inheritance. This is, first and foremost, to keep the calculus simple.
Secondly, there are many different approaches to inheritance and
mixin composition, so that it looks advantageous not to tie the basic
calculus to a specific one. Finally, it seems that the modelization of
inheritance lends itself to a different approach than the basic
calculus. For the latter, we need to prove type safety of the calculus.
One might try to do this also for a calculus with inheritance, but our
experience suggests that this complicates the proofs considerably.  An
alternative approach to inheritance that might work better is to model
it as a form of code-reuse. Starting with an enriched type system with
inheritance, and a translation to the basic calculus, one needs to
show type safety wrt the translation. This might be easier than
to prove type safety wrt reduction.

% TODO: ask: does this last point still makes sense wrt to type safety via logical relations instead of via the standard theorems of preservation and progress?

\section{The DOT Calculus}

\boxfig{dot-one}{The DOT Calculus : Syntax, Reduction, Type Assignment}{
{\bf Syntax}\medskip
\begin{center}    
$\ba{l@{\hspace{0.2mm}}|@{\hspace{0.2mm}}l}
\ba[t]{l@{\hspace{10mm}}l}
x, y, z    & \lindent{\mbox{Variable}} \\
l          & \lindent{\mbox{Value label}}\\
m          & \lindent{\mbox{Method label}}\\[0.2em]
v ::=      & \lindent{\mbox{Value}} \\
\gap x     & \mbox{variable} \\[0.2em]
t ::=      & \lindent{\mbox{Term}} \\
\gap v     & \mbox{value} \\
\gap \new x c t & \mbox{new instance} \\
\gap t.l  & \mbox{field selection} \\
\gap \mapp t m t  & \mbox{method invocation} \\[0.2em]
p ::= & \lindent \mbox{Path} \\
\gap x & \mbox{variable} \\
\gap p.l & \mbox{selection} \\
c ::= T_c \ldefs{\seq{l = v}\;\seq{m(x) = t}} & \lindent{\mbox{Constructor}} \\[0.2em]
\Gamma ::= \seq{x \typ T} & \lindent\mbox{Environment} \\
s      ::= \seq{x \mapsto c} & \lindent\mbox{Store} \\
\ea
&
\ba[t]{l@{\hspace{10mm}}l}
L ::=      & \lindent{\mbox{Type label}} \\
\gap L_c   & \mbox{class label} \\
\gap L_a   & \mbox{abstract type label} \\[0.2em]
S,T,U,V,W ::= & \lindent\mbox{Type}\\
\gap p.L & \mbox{type selection} \\
\gap T \refine z {\seq D} & \mbox{refinement} \\
\gap T \tand T & \mbox{intersection type} \\
\gap T \tor T & \mbox{union type} \\
\gap \Top  & \mbox{top type} \\
\gap \Bot  & \mbox{bottom type} \\[0.2em]
S_c, T_c ::= & \lindent \mbox{Concrete type} \\
\multicolumn{2}{l}{\gap p.L_c ~|~ T_c \refine z {\seq D} ~|~ T_c \wedge T_c  ~|~ \Top} \\[0.2em]
D ::= & \lindent\mbox{Declaration} \\
\gap \Ldecl L S U & \mbox{type declaration} \\
\gap \ldecl l T   & \mbox{value declaration} \\
\gap \mdecl m S U & \mbox{method declaration}
\ea
\ea$
\end{center}
\medskip

\linesep

\begin{multicols}{2}[\judgement{Reduction}{\fbox{$\reduction t s {t'} {s'}$}}]

\infrule[\textsc{msel}]
{y \mapsto T_c \ldefs{\seq{l = v'}\;\seq{m(x)=t}} \in s}
{\reduction {\mapp y {m_i} v} s {\subst v {x_i} {t_i}} s}

\infrule[\textsc{sel}]
{y \mapsto T_c \ldefs{\seq{l = v}\;\seq{m(x)=t}} \in s}
{\reduction {y.l_i} s {v_i} s}

\infax[\textsc{new}]
{\reduction {\new x c t} s t {s \envplus{x \mapsto c}}}

\infrule[\textsc{context}]
{\reduction t s {t'} {s'}}
{\reduction {e[t]} s {e[t']} s'}
\end{multicols}

\hfill {\bf where} evaluation context $\gap e ::= [\,] ~|~ \mapp e m t ~|~ \mapp v m e ~|~ e.l\hspace{2cm}$

\linesep

\begin{multicols}{2}[\judgement{Type Assignment}{\fbox{$\Gamma \ts t \typ T$}}]

\infrule[\textsc{var}]
{x \typ T \in \Gamma}
{\Gamma \ts x \typ T}

\infrule[\textsc{msel}]
{\Gamma \ts t \ni m \typ {S \tfun T} \\
 \Gamma \ts t' \typ T' \spcomma T' \sub S}
{\Gamma \ts \mapp t m {t'} \typ T}

\infrule[\textsc{sel}]
{\Gamma \ts t \ni l \typ T'}
{\Gamma \ts t.l \typ T'}

\infrule[\textsc{new}]
{
c = {T_c \ldefs{\seq{l = v}\;\seq{m(x) = t}}}\\
y \notin \fn(T') \\
\Gamma \ts T_c \wf \\
\Gamma \ts T_c \expand_y \seq{\Ldecl L S U},\seq{\ldecl l V},\seq{\mdecl m T W} \\
\Gamma \envplus{y: T_c} \ts \seq{S \sub U}\\
\Gamma \envplus{y: T_c} \ts \seq{v \typ V'} \spcomma \seq{V' \sub V}\\
\Gamma \envplus{y: T_c} \ts \seq{{T_i} \wfe}\\
\seq{\Gamma \envplus{y: T_c} \envplus{x_i: T_i} \ts t_i \typ {W_i}' \spcomma {W_i}' \sub {W_i}}\\
\Gamma \envplus{y: T_c} \ts  {t'} \typ T'}
{\Gamma \ts \new y c {t'} \typ T'}

\end{multicols}
} % END dot-one

The DOT calculus is a simple system of dependent
object-types. \figref{dot-one} gives its syntax, reduction rules,
and type assignment rules.

\subsection*{Notation} We use standard notational conventions for
sets. The notation $\seq{X}$ denotes a set of elements $X$. Given a
such a set $\seq X$ in a typing rule, $X_i$ denotes an arbitrary
element of $X$. 
%The $\uplus$ operator extends a set of bindings. It is required that the added binding does not
%introduce a variable which is already bound in the base-set.
We use an
abbreviation for preconditions in typing judgements. Given an
environment $\Gamma$ and some predicates $P$ and $Q$, the condition $\Gamma \ts P \spcomma Q$
is a shorthand for the two conditions $\Gamma \ts P$ and $\Gamma \ts Q$.

\subsection*{Syntax}

There are four alphabets: Variable names $x$, $y$, $z$ are freely
alpha-renamable. They occur as parameters of lambda abstractions, as
binders for objects created by \verb@new@-expressions, and as self
references in refinements. Value labels $l$ denote fields in objects,
which are bound to values at run-time. Similarly, method labels $m$
denote methods in objects. Type labels $L$ denote type members of
objects. Type labels are further separated into labels for abstract
types $L_a$ and labels for classes $L_c$. It is assumed that in each
program every class label $L_c$ is declared at most once.

We assume that the label alphabets $l$, $m$ and $L$ are finite. This is
not a restriction in practice, because one can include in these 
alphabets every label occurring in a given program.

The terms $t$ in DOT consist of variables $x$, $y$, $z$, field
selections $t.l$, method invocations $t.m(t)$ and object creation
expressions $\new y c {t'}$ where $c$ is a constructor $T_c \ldefs{\seq{l
    = v}\;\seq{m(x) = t}}$. The latter binds a variable $y$ to a new
instance of type $T_c$ with fields $\seq l$ initialized to values
$\seq v$ and methods $\seq m$ initialized to methods of one parameter
$\seq{x}$ and body $\seq{t}$.  The scope of $y$ extends through the term
${t'}$.

Two subclasses of terms are values $v$, which consist of just
variables, and paths $v$ which consist of just variables and field
selections.

The types in DOT are denoted by letters $S$, $T$, $U$, or $V$. They consist of the following:
\begin{itemize}
\item[-] Type selections $p.L$, which denote the type member $L$ of path $p$.
\item[-] Refinement types $T \refine z {\seq D}$, which refine a type $T$ by a set of declarations $D$.
         The variable $z$ refers to the ``self''-reference of the type. Declarations can refer to
         other declarations in the same type by selecting from $z$.
\item[-] Type intersections $T \tand T'$, which carry the declarations of members present in either $T$ or $T'$.
\item[-] Type unions $T \tor T'$, which carry only the declarations of members present in both $T$ and $T'$.
\item[-] A top type $\Top$, which corresponds to an empty object.
\item[-] A bottom type $\Bot$, which represents a non-terminating computation.
\end{itemize}
A subset of types $T_c$ are called {\em concrete types}. These are type selections
$p.L_c$ of class labels,
the top type $\Top$, intersections of concrete types, and refinements $T_c \refine z {\seq D}$ of concrete types. Only concrete types are allowed in constructors $c$.

There are only three forms of declarations in DOT, which are both part
of refinement types.  A value declaration $\ldecl l T$ introduces a
field with type $T$.  A method declaration $\mdecl m S U$ introduces a
method with parameter of type $S$ and result of type $U$. A type
declaration $\Ldecl L S U$ introduces a type member $L$ with a lower
bound type $S$ and an upper bound type $U$. There are no type aliases,
but a type alias can be simulated by a type declaration $\Ldecl L T T$
where the lower bound and the upper bound are the same type $T$.

\boxfig{dot-decls}{The DOT Calculus : Declaration Lattice}{

\bda{lcl@{\gap}l}
    
      \dom(\seq D \tand \seq {D'}) &~=~& \dom(\seq{D}) \cup \dom(\seq{D'}) \\
      \dom(\seq D \tor \seq {D'}) &=& \dom(\seq{D}) \cap \dom(\seq{D'}) \\[0.5em]
      (D \tand D')(L) &=&
        \Ldecl L {(S \tor S')} {(U \tand U')} & \mbox{if~} (\Ldecl L S U) \in \seq{D} \;\mbox{and}\; (\Ldecl L {S'} {U'}) \in \seq{D'} \\
        &=& D(L) & \mbox{if~} L \notin \dom(\seq{D'}) \\
        &=& D'(L) & \mbox{if~} L \notin \dom(\seq{D}) \\
      (D \tand D')(m) &=&
        \mdecl m {(S \tor S')} {(U \tand U')} & \mbox{if~} (\mdecl m S U) \in \seq{D} \;\mbox{and}\; (\mdecl m {S'} {U'}) \in \seq{D'} \\
        &=& D(m) & \mbox{if~} m \notin \dom(\seq{D'}) \\
        &=& D'(m) & \mbox{if~} m \notin \dom(\seq{D}) \\
      (D \tand D')(l) &=&
        \ldecl l {T \tand T'} & \mbox{if~} (\ldecl l T) \in \seq{D} \;\mbox{and}\; (\ldecl l {T'}) \in \seq{D'} \\
        &=& D(l) & \mbox{if~} l \notin \dom(\seq{D'}) \\
        &=& D'(l) & \mbox{if~} l \notin \dom(\seq{D}) \\[0.5em]
      (D \tor D')(L) &=&
        \Ldecl L {(S \tand S')} {(U \tor U')} & \mbox{if~} (\Ldecl L S U) \in \seq{D} \;\mbox{and}\; (\Ldecl L {S'} {U'}) \in \seq{D'} \\
      (D \tor D')(m) &=&
        \mdecl m {(S \tand S')} {(U \tor U')} & \mbox{if~} (\mdecl m S U) \in \seq{D} \;\mbox{and}\; (\mdecl m {S'} {U'}) \in \seq{D'} \\
      (D \tor D')(l) &=&
        \ldecl l {T \tor T'} & \mbox{if~} (\ldecl l T) \in \seq{D} \;\mbox{and}\; (\ldecl l {T'}) \in \seq{D'} 
    \eda

Sets of declarations form a lattice with the given meet $\wedge$ and
join $\vee$, the empty set of declatations as the top element, and the
bottom element $\seq{D_\Bot}$, Here $\seq{D_\Bot}$ is the set of
declarations that contains for every term label $l$ the declaration
$\ldecl l \Bot$, for every type label $L$ the declaration $\Ldecl L
\Top \Bot$ and for every method label $m$ the declaration $\mdecl m
\Top \Bot$.
}

Every field or type label can be declared only once in a set of
declarations $\seq D$. A set of declarations can hence be seen as a map from
labels to their declarations.  Meets $\wedge$ and joins $\vee$ on sets of
declarations are defined in ~\figref{dot-decls}.

\subsection*{Reduction rules}

Reduction rules $\reduction t s {t'} {s'}$ in DOT rewrite pairs of
terms $t$ and stores $s$, where stores map variables to constructors.
There are three main reduction rules: Rule (\textsc{msel}) rewrites a
method invocation $y {m_i} v$ by retrieving the corresponding method
definition from the store, and performing a substitution of the
argument for the parameter in the body. Rule (\textsc{sel}) rewrites a
field selection $x.l$ by retrieving the corresponding value from the
store. Rule (\textsc{new}) rewrites an object creation $\new x c t$ by
placing the binding of variable $x$ to constructor $c$ in the store
and continuing with term $t$.  These reduction rules can be applied
anywhere in a term where the hole $[\,]$ of an evaluation context $e$
can be situated.

\subsection*{Type assignment rules}

The last part of \figref{dot-one} presents rules for type
assignment.  

Rules (\textsc{sel}) and (\textsc{msel}) type field selections and
method invocations by means of an auxiliary membership relation $\ni$,
which determines whether a given term contains a given declaration as
one of its members. The membership relation is defined in
\figref{dot-two} and is further explained below.

The last rule, (\textsc{new}), assigns types to object creation
expressions. It is the most complex of DOT's typing rules.  To
type-check an object creation $\new y {T_c \ldefs {\seq{l =
      v}\;\seq{m(x) = t}}} t'$, one verifies first that the type $T_c$
is well-formed (see \figref{dot-wf} for a definition of
well-formedness).  One then determines the set of all declarations
that this type carries, using the expansion relation $\expand$ defined
in \figref{dot-two}.  Every type declaration $\Ldecl L S U$ in
this set must be realizable, i.e.\ its lower bound $S$ must be a
subtype of its upper bound $U$.  Every field declaration $\ldecl l V$
in this set must have a corresponding initializing value of $v$ of
type $V$.  These checks are made in an environment which is extended
by the binding $y: T_c$. In particular this allows field values that
recurse on ``self'' by referring to the bound variable $x$. Similarly,
every method declaration $\mdecl m T W$ must have a corresponding
initializing method definition $m(x) = t$. The parameter type $T$ must
be $\wfe$ (well-formed and expanding; see \figref{dot-wf}), and
the body $t$ must type check to $W$ in an environment extended by the
bindings $y : T_c$ and $x : T$.

Instead of adding a separate subsumption rule, subtyping is expressed
by preconditions in rules (\textsc{msel}) and (\textsc{new}).

\boxfig{dot-two}{The DOT Calculus : Membership and Expansion}{
    \begin{multicols}{2}[\judgement{Membership}{\fbox{$\Gamma \ts t \ni D$}}]

      \infrule[\textsc{path-$\ni$}]
      {\Gamma \ts p \typ T \spcomma T \expand_z \seq D}
      {\Gamma \ts p \ni \subst p z {D_i}}

      \infrule[\textsc{term-$\ni$}]
      {z \not\in \fn(D_i) \andalso \Gamma \ts t \typ T \spcomma T \expand_z \seq D}
      {\Gamma \ts t \ni D_i}
   \end{multicols}

   \linesep

    \begin{multicols}{2}[\judgement{Expansion}{\fbox{$\Gamma \ts T \expand_z \seq D$}}]  

      \infrule[\textsc{rfn-$\expand$}]
      {\Gamma \ts T \expand_z {\seq D'}}
      {\Gamma \ts T \refine z {\seq D} \expand_z \seq {D'} \tand \seq D}

      \infrule[\textsc{$\tand$-$\expand$}]
      {\Gamma \ts T_1 \expand_z {\seq D_1} \spcomma T_2 \expand_z {\seq D_2}}
      {\Gamma \ts T_1 \tand T_2 \expand_z {\seq D_1 \tand \seq D_2}}

      \infax[\textsc{$\Top$-$\expand$}]
      {\Gamma \ts \Top \expand_z \{\}}

      \infrule[\textsc{tsel-$\expand$}]
      {\Gamma \ts p \ni \Ldecl L S U \spcomma U \expand_z \seq D}
      {\Gamma \ts p.L \expand_z \seq D}

      \infrule[\textsc{$\tor$-$\expand$}]
      {\Gamma \ts T_1 \expand_z {\seq D_1} \spcomma T_2 \expand_z {\seq D_2}}
      {\Gamma \ts T_1 \tor T_2 \expand_z {\seq D_1 \tor \seq D_2}}

      \infax[\textsc{$\Bot$-$\expand$}]
      {\Gamma \ts \Bot \expand_z \seq{D_\Bot}}
    \end{multicols}

} % END dot-two

\subsection*{Membership}

\figref{dot-two} presents typing rules for membership and
expansion.  The membership judgement $\Gamma \ts t \ni D$ states that
in environment $\Gamma$ a term $t$ has a declaration $D$ as a member.
There are different rules for paths and general terms.  Rule
(\textsc{path-$\ni$}) applies to paths $p$ that have a refinement type
$T \refine z {\seq D}$ as their type. Members of $p$ are then all
definitions $D$ in the refinement, where any use of the self reference
$z$ is replaced by the path $p$ itself.  Rule (\textsc{and-$\ni$})
allows to merge two member definitions (of the same label) by
conjoining them with $\tand$. Rule (\textsc{term-$\ni$}) establishes
the members of general terms $t$ of type $T$ by introducing a dummy
variable $z$ of type $T$ and then using the rules for path membership
on $z$. It must hold that $z$ itself does not form part of the
resulting member $D$.

\subsection*{Expansion}
      
The expansion relation $\expand$ is needed to typecheck the complete
set of declarations carried by a concrete type that is used in a
\textbf{new}-expression. Expansion is also used in subtyping
refinements on the right (see \figref{dot-sub}).

Rule (\textsc{rfn-$\expand$}) states that a refinement type $T
\expand_z {\seq D}$ expands to the conjunction of the expansion $D'$
of $T$ and the newly added declarations $D$. Rule
(\textsc{tsel-$\expand$}) states that a type selection $p.L$ carries
the same declarations as the upper bound $U$ of $L$ in $T$.  Rules
($\tand$-$\expand$) and ($\tor$-$\expand$) states that expansion
distributes through meets and joins.  Rule (\textsc{$\top$-$\expand$})
states that the top type $\top$ expands to the empty set. Rule
(\textsc{$\bot$-$\expand$}) states that the bottom type $\bot$ expands
to the bottom element $\seq{D_\Bot}$ of the lattice of sets of
declarations (recall \figref{dot-decls}).

\boxfig{dot-sub}{The DOT Calculus : Subtyping and Declaration Subsumption}{

    \begin{multicols}{2}[\judgement{Subtyping}{\fbox{$\Gamma \ts S \sub T$}}]

%      \infrule[\textsc{trans}]
%      {\Gamma \ts S \sub T \spcomma T \sub U}
%      {\Gamma \ts S \sub U}

      \infrule[\textsc{refl}]
      {\Gamma \ts T \wfe}
      {\Gamma \ts T \sub T}

      \infrule[\textsc{$\sub$-rfn}]
      {\Gamma \ts {T \refine z {\seq D}} \wfe
       \spcomma S \sub T \spcomma S \expand_z \seq{D'} \\
       \Gamma \envplus{z: S} \ts \seq{D' \sub D}}
      {\Gamma \ts S \sub T \refine z {\seq D}}

      \infrule[\textsc{$\sub$-tsel}]
      {\Gamma \ts p \ni \Ldecl L S U \spcomma S \sub U \spcomma S' \sub S }
      {\Gamma \ts S' \sub p.L}

      \infrule[\textsc{$\sub$-$\tand$}]
      {\Gamma \ts T \sub T_1 \spcomma T \sub T_2}
      {\Gamma \ts T \sub T_1 \tand T_2}

      \infrule[\textsc{$\sub$-$\tor_1$}]
      {\Gamma \ts T_2 \wfe \spcomma T \sub T_1}
      {\Gamma \ts T \sub T_1 \tor T_2}

      \infrule[\textsc{$\sub$-$\tor_2$}]
      {\Gamma \ts T_1 \wfe \spcomma T \sub T_2}
      {\Gamma \ts T \sub T_1 \tor T_2}
      
      %%%%%%%%%%%%%%%%%%%%%%%%%%%%%%%%%

      \infrule[\textsc{$\sub$-$\Top$}]
      {\Gamma \ts T \wfe}
      {\Gamma \ts T \sub \Top}

      \infrule[\textsc{$\Bot$-$\sub$}]
      {\Gamma \ts T \wfe}
      {\Gamma \ts \Bot \sub T}

      \infrule[\textsc{rfn-$\sub$}]
      {\Gamma \ts {T \refine z {\seq D}} \wfe
       \spcomma T \sub T'}
      {\Gamma \ts T \refine z {\seq D} \sub T'}

      \infrule[\textsc{tsel-$\sub$}]
      {\Gamma \ts p \ni \Ldecl L S U \spcomma S \sub U \spcomma U \sub U'}
      {\Gamma \ts p.L \sub U'}

      \infrule[\textsc{$\tor$-$\sub$}]
      {\Gamma \ts T_1 \sub T \spcomma T_2 \sub T}
      {\Gamma \ts T_1 \tor T_2 \sub T}

      \infrule[\textsc{$\tand_1$-$\sub$}]
      {\Gamma \ts T_2 \wfe \spcomma T_1 \sub T}
      {\Gamma \ts T_1 \tand T_2 \sub T}

      \infrule[\textsc{$\tand_2$-$\sub$}]
      {\Gamma \ts T_1 \wfe \spcomma T_2 \sub T}
      {\Gamma \ts T_1 \tand T_2 \sub T}

      \end{multicols}

    \linesep

    \begin{multicols}{2}[\judgement{Declaration subsumption}{\fbox{$\Gamma \ts D \sub D'$}}]

    \infrule[\textsc{tdecl-$\sub$}]
            {\Gamma \ts S' \sub S \spcomma T \sub T'}
            {\Gamma \ts (\Ldecl L S T) \sub (\Ldecl L {S'} {T'})}

    \infrule[\textsc{vdecl-$\sub$}]
            {\Gamma \ts T \sub T'}
            {\Gamma \ts (\ldecl l T) \sub (\ldecl l {T'})}

    \infrule[\textsc{mdecl-$\sub$}]
            {\Gamma \ts S' \sub S \spcomma T \sub T'}
            {\Gamma \ts (\mdecl m S T) \sub (\mdecl m {S'} {T'})}

    \end{multicols}
} % END dot-sub

\subsection*{Subtyping}

\figref{dot-sub} defines the subtyping judgement $\Gamma \ts S
\sub T$ which states that in environment $\Gamma$ type $S$ is a
subtype of type $T$. Subtyping is regular with respect to $\wfe$: if
type $S$ is a subtype of type $T$, then $S$ and $T$ are well-formed
and expanding. Though this regularity limits our calculus to
$\wfe$-types, this limitation allows us to show that subtyping is
transitive, as discussed in section~\ref{dot-preservation}.

\boxfig{dot-wf}{The DOT Calculus : Well-Formedness}{

       \begin{multicols}{2}[\judgement{Well-formed types}{\fbox{$\Gamma \ts T \wf$}}]

      \infrule[\textsc{rfn-wf}]
      {\Gamma \ts T \wf \\ 
       \Gamma \envplus {z: T \refine z {\seq D}} \ts \seq {D \wf}}
      {\Gamma \ts T \refine z {\seq D} \wf}

      \infrule[\textsc{tsel-wf$_1$}]
      {\Gamma \ts p \ni \Ldecl L S U \spcomma S \wf \spcomma U \wf}
      {\Gamma \ts p.L \wf}

      \infrule[\textsc{$\tand$-wf}]
      {\Gamma \ts T \wf \spcomma T' \wf}
      {\Gamma \ts T \tand T' \wf}

      %%%%%%%%%%%%%%%%%%%%%%%%%%%

      \infax[\textsc{$\Top$-wf}]
      {\Gamma \ts \Top \wf}

      \infax[\textsc{$\Bot$-wf}]
      {\Gamma \ts \Bot \wf}

      \infrule[\textsc{tsel-wf$_2$}]
      {\Gamma \ts p \ni \Ldecl L \Bot U}
      {\Gamma \ts p.L \wf}

      \infrule[\textsc{$\tor$-wf}]
      {\Gamma \ts T \wf \spcomma T' \wf}
      {\Gamma \ts T \tor T' \wf}

    \end{multicols}

    \linesep

    \begin{multicols}{2}[\judgement{Well-formed declarations}{\fbox{$\Gamma \ts D \wf$}}]
      \infrule[\textsc{tdecl-wf}]
      {\Gamma \ts S \wf \spcomma U \wf}
      {\Gamma \ts \Ldecl L S U \wf}

      \infrule[\textsc{vdecl-wf}]
      {\Gamma \ts T \wf}
      {\Gamma \ts \ldecl l T \wf}

      \infrule[\textsc{mdecl-wf}]
      {\Gamma \ts S \wf \spcomma U \wf}
      {\Gamma \ts \mdecl m S U \wf}

    \end{multicols}

    \linesep

    \begin{multicols}{2}[\judgement{Well-formed and expanding types}{\fbox{$\Gamma \ts T \wfe$}}]
      \infrule[\textsc{wfe}]
      {\Gamma \ts T \wf \spcomma T \expand_z \seq{D}}
      {\Gamma \ts T \wfe}
    \end{multicols}
}

\section{Counterexamples to Preservation}\label{dot-preservation}

\section{Type Safety via Logical Relations}\label{dot-type-safety}

\section{Conclusion}\label{conclusion}

%\appendix
%\section{Appendix Title}
%\acks

\bibliographystyle{abbrvnat}
\bibliography{dot}

\end{document}
